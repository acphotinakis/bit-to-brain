\documentclass[12pt]{article}
\usepackage[utf8]{inputenc}
\usepackage{amsmath, amssymb}
\usepackage{xcolor}
\usepackage{geometry}
\usepackage{hyperref}
\usepackage{fancyhdr}
\usepackage{enumitem}
\usepackage{minted}
\usepackage{booktabs}
\usepackage{graphicx}
\usepackage{tikz}
\usepackage{caption}
\usetikzlibrary{shapes, arrows, positioning}

% -------------------------------
% Page and link settings
% -------------------------------
\geometry{margin=1in}
\hypersetup{
    colorlinks=true,
    linkcolor=blue,
    urlcolor=cyan
}

% -------------------------------
% Custom Commands & Metadata
% -------------------------------
\newcommand{\TOPICTITLE}{Application Layer - Additional Clarity Keywords}

\pagestyle{fancy}
\fancyhf{}
\fancyhead[L]{\textbf{\TOPICTITLE}}
\fancyhead[R]{\thepage}
\setlength{\headheight}{15pt}

\title{\TOPICTITLE\\\large Study-Ready Notes}
\author{Compiled by Andrew Photinakis}
\date{\today}

% -------------------------------
% Begin Document
% -------------------------------
\begin{document}

\maketitle
\tableofcontents
\newpage

\section{Ports}

\subsection{What is a Port?}
\begin{enumerate}
    \item A port is like a doorway into a computer for network communication.
    \item Every device on a network (like your laptop, a web server, or a phone) has:
          \begin{itemize}
              \item A unique IP address, which identifies the device.
              \item Multiple ports, which identify specific applications (processes) running on that device.
          \end{itemize}
    \item Analogy: Think of the IP address as the street address of an apartment building, and the port numbers as the apartment numbers inside it.
\end{enumerate}

\subsection{Why we need ports?}
\begin{enumerate}
    \item Many programs can use the network at once:
          \begin{itemize}
              \item Your web browser.
              \item Your email client.
              \item A game you are playing online.
              \item A video call using Zoom or Skype.
          \end{itemize}
    \item All of these share the same IP address (your computer's address on the network), but each communicates using a different port number so the OS knows which process should get each incoming packet.
    \item Without ports, your computer would not know whether an incoming packet was meant for your browser or your email app.
\end{enumerate}

\subsection{Structure: IP + Port = Socket}
\begin{enumerate}
    \item Each network connection is identified by a socket pair:
          \begin{minted}{bash}
(source IP, source port) → (destination IP, destination port)
          \end{minted}
    \item Example (you visiting a website):
          \begin{itemize}
              \item \begin{minted}{bash}
(192.168.1.10, 49523) → (128.119.245.12, 80)
                    \end{minted}
              \item Your computer (client) uses a temporary port like 49523 assigned by your OS.
              \item The web server listens on port 80 for HTTP requests.
          \end{itemize}
\end{enumerate}

\subsection{Port Number Ranges}
\begin{center}
    \begin{tabular}{|c|c|c|}
        \hline
        \textbf{Range} & \textbf{Description}                & \textbf{Examples}                   \\
        \hline
        0--1023        & Well-known ports (assigned by IANA) & HTTP 80, HTTPS 443, SMTP 25, DNS 53 \\
        1024--49151    & Registered ports (specific apps)    & MySQL 3306, NFS 2049                \\
        49152--65535   & Dynamic/private ports (temporary)   & Client ephemeral ports              \\
        \hline
    \end{tabular}
\end{center}

\subsection{Common Port Numbers}
\begin{center}
    \begin{tabular}{|c|c|c|}
        \hline
        \textbf{Service}       & \textbf{Protocol} & \textbf{Port} \\
        \hline
        HTTP                   & TCP               & 80            \\
        HTTPS (Secure HTTP)    & TCP               & 443           \\
        FTP (File Transfer)    & TCP               & 21            \\
        SMTP (Email sending)   & TCP               & 25            \\
        IMAP (Email retrieval) & TCP               & 143           \\
        DNS                    & UDP/TCP           & 53            \\
        SSH (Secure Shell)     & TCP               & 22            \\
        Telnet                 & TCP               & 23            \\
        \hline
    \end{tabular}
\end{center}

\subsection{Ports and Protocols (TCP vs UDP)}
\begin{center}
    \begin{tabular}{|c|c|c|}
        \hline
        \textbf{Feature} & \textbf{TCP Port}         & \textbf{UDP Port}      \\
        \hline
        Connection-based & Yes                       & No                     \\
        Reliability      & Guaranteed                & Best effort            \\
        Use case         & Web, Email, File Transfer & DNS, Streaming, Gaming \\
        \hline
    \end{tabular}
\end{center}

\subsection{Ports and Security}
\begin{enumerate}
    \item Ports can act as entry points for attacks.
    \item Firewalls are configured to block or allow specific ports.
    \item Port scanning tools (like \texttt{nmap}) are used to identify open or vulnerable ports.
    \item Example: A secure web server only opens port 443 (HTTPS) instead of 80, ensuring all traffic is encrypted.
\end{enumerate}

\subsection{Real Example: Visiting a Website}
\begin{enumerate}
    \item Your browser creates a TCP connection.
    \item The OS assigns an ephemeral (temporary) port on your machine, e.g., 54321.
    \item It sends a packet to:
          \begin{minted}{bash}
destination IP: 128.119.245.12
destination port: 80
          \end{minted}
    \item The server receives it on port 80, where its web server software (like Apache) is listening.
    \item The server replies from (80) → (54321).
    \item When done, the connection closes and your port 54321 becomes free again.
\end{enumerate}

\subsection{Summary}
\begin{center}
    \begin{tabular}{|c|p{8cm}|}
        \hline
        \textbf{Concept} & \textbf{Explanation}                                 \\
        \hline
        Port             & A number identifying a process or service on a host. \\
        IP + Port        & Together identify a specific communication endpoint. \\
        Server port      & Fixed, well-known (e.g., 80, 443).                   \\
        Client port      & Temporary, dynamically assigned.                     \\
        Socket pair      & Defines one full connection between two hosts.       \\
        Firewall use     & Controls access to ports for security.               \\
        \hline
    \end{tabular}
\end{center}

\subsection{Quick Recap}
\begin{itemize}
    \item Ports separate traffic for multiple applications on one device.
    \item Port numbers range from 0--65535.
    \item Well-known ports are reserved for common services.
    \item Clients use ephemeral ports; servers use fixed ones.
    \item Both TCP and UDP use port numbers, but handle connections differently.
\end{itemize}

\section{Ports in Computer Networks}

\subsection{Introduction}
In computer networks, a \textbf{port} is a logical endpoint for communication that allows multiple networked applications to coexist on a single device. While an \textbf{IP address} identifies \textit{which device} on the network to reach, a \textbf{port number} specifies \textit{which process or service} within that device should receive the data. This combination of IP address and port number forms a \textbf{socket} --- the foundation of process-to-process communication across the Internet.

\textcolor{blue}{[Summary: Ports serve as numbered entry points that allow multiple applications to share the same network connection on a device.]}

\subsection{Keyword Breakdown}
\begin{itemize}[label=\textbullet]
    \item \textbf{Port Number:} A 16-bit integer (0–65535) identifying a specific application or process.
    \item \textbf{Socket:} The pairing of an IP address and port number (e.g., 192.168.1.10:443).
    \item \textbf{Well-Known Ports:} Reserved for standard Internet services (0–1023).
    \item \textbf{Registered Ports:} Assigned by IANA to specific applications (1024–49151).
    \item \textbf{Dynamic or Private Ports:} Used temporarily by client applications (49152–65535).
\end{itemize}

\textcolor{orange}{[Mnemonic: ``W-R-D'' — Well-known, Registered, Dynamic — helps recall the three port ranges.]}

\subsection{Analogy: Apartments in a Building}
An IP address is like the \textit{street address} of an apartment building, and ports are the \textit{apartment numbers}. Data arriving at the building (IP address) must know which apartment (port) to go to. For instance, a web server might live in apartment 80, while an email server lives in apartment 25.

\textcolor{blue}{[Summary: The port number directs network messages to the correct application, much like an apartment number directs mail within a building.]}

\subsection{How Ports Function in Communication}
Every Internet connection involves two endpoints, each identified by a socket:
\[
    \text{Socket Pair: } ( \text{Source IP, Source Port, Destination IP, Destination Port} )
\]

When a client (browser) requests a webpage:
\begin{enumerate}
    \item The browser selects a random \textbf{source port} (e.g., 51820).
    \item The web server listens on a known \textbf{destination port} (e.g., 443 for HTTPS).
    \item The request is sent as:
          \[
              192.168.1.5:51820 \rightarrow 93.184.216.34:443
          \]
    \item The server’s response is sent back along the same path, using the port numbers to ensure data reaches the right application.
\end{enumerate}

\textcolor{blue}{[Summary: Ports ensure that each networked process on a host receives the correct data among multiple concurrent communications.]}

\subsection{Port Number Categories}

\begin{table}[h!]
    \centering
    \begin{tabular}{@{}lll@{}}
        \toprule
        \textbf{Range} & \textbf{Type}           & \textbf{Example Usage}            \\ \midrule
        0–1023         & Well-Known Ports        & HTTP (80), HTTPS (443), SMTP (25) \\
        1024–49151     & Registered Ports        & MySQL (3306), PostgreSQL (5432)   \\
        49152–65535    & Dynamic / Private Ports & Temporary client connections      \\ \bottomrule
    \end{tabular}
    \caption{Port number categories and examples.}
\end{table}

\textcolor{blue}{[Summary: Port numbers are divided into standardized ranges to manage global consistency and avoid conflicts.]}

\subsection{TCP vs. UDP Ports}
Both TCP and UDP protocols use port numbers, but for different purposes:

\begin{table}[h!]
    \centering
    \begin{tabular}{@{}llll@{}}
        \toprule
        \textbf{Protocol} & \textbf{Type}       & \textbf{Example Port} & \textbf{Description}                           \\ \midrule
        TCP               & Connection-Oriented & 80 (HTTP)             & Reliable data delivery via acknowledgment      \\
        UDP               & Connectionless      & 53 (DNS)              & Faster transmission with no delivery guarantee \\ \bottomrule
    \end{tabular}
    \caption{Comparison of TCP and UDP port usage.}
\end{table}

\textcolor{blue}{[Summary: TCP ports manage reliable, ordered communication; UDP ports handle faster, simpler message delivery.]}

\subsection{Common Port Numbers in Practice}

\begin{table}[h!]
    \centering
    \begin{tabular}{@{}lll@{}}
        \toprule
        \textbf{Service} & \textbf{Protocol} & \textbf{Port} \\ \midrule
        HTTP             & TCP               & 80            \\
        HTTPS            & TCP               & 443           \\
        FTP              & TCP               & 21            \\
        SSH              & TCP               & 22            \\
        DNS              & UDP/TCP           & 53            \\
        SMTP             & TCP               & 25            \\
        POP3             & TCP               & 110           \\
        IMAP             & TCP               & 143           \\
        DHCP             & UDP               & 67, 68        \\ \bottomrule
    \end{tabular}
    \caption{Commonly used port numbers in networking.}
\end{table}

\textcolor{orange}{[Mnemonic: ``2-1-2-2-5-3-8-6'' pattern for FTP (21), SSH (22), SMTP (25), DNS (53), POP3 (110), IMAP (143), HTTPS (443) helps recall key ports.]}

\subsection{Visual Diagram Description}
\begin{quote}
    Imagine a diagram showing:
    \begin{itemize}
        \item A client on the left labeled with ``Source IP: 192.168.1.5, Port: 51820''.
        \item A server on the right labeled ``Destination IP: 93.184.216.34, Port: 443 (HTTPS)''.
        \item Arrows between them representing TCP segments or UDP datagrams.
    \end{itemize}
    This visualization helps reinforce how port numbers map communication between specific processes on each host.
\end{quote}

\textcolor{blue}{[Summary: Visualizing sockets clarifies how each connection is identified by IP and port on both ends.]}

\subsection{Integration with the Broader Network System}
Ports operate within the \textbf{Transport Layer} (Layer 4 of the OSI model) but are essential to the \textbf{Application Layer} (Layer 7) where specific network services reside. The transport layer (e.g., TCP/UDP) uses port numbers to multiplex multiple application streams across a single IP connection.

\textcolor{blue}{[Summary: Ports bridge the gap between transport mechanisms and application processes, enabling simultaneous communication across many services.]}

\subsection{Key Takeaways}

\textcolor{blue}{[Summary: Ports are numerical identifiers that distinguish network services on a host, allowing simultaneous communication over shared IP connections. They are crucial for process-to-process data delivery in TCP/IP networks.]}

\textcolor{orange}{[Mnemonic: ``IP = device, Port = program'' — IP locates the machine, Port locates the process.]}

\textcolor{red}{[Exam Questions:
            (1) Explain the relationship between IP addresses, ports, and sockets.
            (2) Differentiate between well-known, registered, and dynamic ports.
            (3) Compare TCP and UDP port usage and reliability mechanisms.]}




\section{Ports in Computer Networks}

\subsection{Core Definitions}
\begin{itemize}
    \item \textbf{Port:} Numerical identifier used to direct network traffic to the correct application or process.
    \item \textbf{Socket:} Combination of IP address + port number; uniquely identifies a communication endpoint.
    \item \textbf{Well-Known Ports:} Ports 0--1023 reserved for standard services (e.g., HTTP=80, HTTPS=443).
    \item \textbf{Ephemeral Ports:} Temporary ports (1024--65535) assigned by client for short-lived sessions.
\end{itemize}

\subsection{Keyword Breakdown}
\begin{itemize}
    \item \textbf{IP Address:} Device location on network.
    \item \textbf{Port Number:} Specific application/service identifier on a device.
    \item \textbf{TCP vs UDP:}
          \begin{itemize}
              \item TCP: Connection-oriented, reliable.
              \item UDP: Connectionless, faster, no delivery guarantee.
          \end{itemize}
    \item \textbf{Multiplexing:} Multiple applications share one IP using different ports.
\end{itemize}

\subsection{Step-by-Step Function}
\begin{enumerate}
    \item Client chooses ephemeral port \& sends request to server IP + well-known port.
    \item Server receives packet, inspects destination port.
    \item Packet routed to correct application/service.
    \item Response sent back to client socket (IP + ephemeral port).
    \item Communication continues until session ends.
\end{enumerate}

\subsection{Examples \& Applications}
\begin{itemize}
    \item HTTP: Port 80 → Web pages
    \item HTTPS: Port 443 → Secure web pages
    \item FTP: Ports 20/21 → File transfers
    \item SSH: Port 22 → Secure remote login
    \item DNS: Port 53 → Domain name resolution
\end{itemize}

\subsection{Comparisons / Contrasts}
\begin{itemize}
    \item \textbf{Port vs IP:} IP = device, Port = application on device.
    \item \textbf{TCP vs UDP Ports:}
          \begin{itemize}
              \item TCP: Reliable, ordered, connection-based.
              \item UDP: Fast, unordered, connectionless.
          \end{itemize}
    \item Well-Known vs Ephemeral:
          \begin{itemize}
              \item Well-Known: Fixed for standard services.
              \item Ephemeral: Dynamic, temporary for clients.
          \end{itemize}
\end{itemize}

\subsection{Analogies}
\begin{itemize}
    \item IP Address = House address
    \item Port = Specific room in the house
    \item Socket = House + Room (full delivery location)
\end{itemize}

\subsection{Visual / Diagram Description}
\begin{itemize}
    \item Server = building, rooms = ports
    \item Client sends packets → addressed to room number (port)
    \item Responses follow reverse path to client's ephemeral port
\end{itemize}

\subsection{Concept Integration}
\begin{itemize}
    \item Works at Transport Layer (OSI Layer 4)
    \item Allows multiple applications to share a single IP
    \item Essential for TCP/IP networking and client-server models
\end{itemize}

\subsection{Summary \& Study Aids}
\textcolor{blue}{[Summary: Ports identify applications on a host, enabling organized network communication with IP addresses and sockets.]}

\textcolor{orange}{[Mnemonic: IP = House, Port = Room, Socket = House + Room]}

\textcolor{red}{[Exam Questions:
            \begin{enumerate}
                \item Difference between well-known and ephemeral ports.
                \item Role of socket in network communication.
                \item Examples of services and default ports.
                \item Compare TCP and UDP port behavior.
            \end{enumerate}
        ]}
\end{document}