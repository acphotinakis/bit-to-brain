\documentclass[12pt]{article}
\usepackage[utf8]{inputenc}
\usepackage{amsmath, amssymb}
\usepackage{xcolor}
\usepackage{geometry}
\usepackage{hyperref}
\usepackage{fancyhdr}
\usepackage{enumitem}
\usepackage{minted} % Code highlighting
\usepackage{booktabs} % Clean tables
\usepackage{tikz} % Optional for concept maps

\geometry{margin=1in}
\hypersetup{colorlinks=true, linkcolor=blue, urlcolor=cyan}

\pagestyle{fancy}
\fancyhf{}
\fancyhead[L]{\textbf{\TOPICTITLE}}
\fancyhead[R]{\thepage}

% -------------------------------
% Topic Metadata
% -------------------------------
\newcommand{\TOPICTITLE}{Application Layer: Socket Programming with UDP and TCP}
\title{\TOPICTITLE\\\large Study-Ready Notes}
\author{Compiled by Andrew Photinakis}
\date{\today}

\setlength{\headheight}{15pt}

\begin{document}
\maketitle
\tableofcontents
\newpage

% This LaTeX file should be saved at: computer_networks/chapter02/application_layer.tex

\section{Socket Programming Fundamentals}

\subsection{Socket Definition and Purpose}

\begin{itemize}
    \item \textbf{Socket}: A door between application process and end-to-end transport protocol
    \item Acts as an interface between application layer and transport layer
    \item Application developer controls the application process
    \item Operating system controls the socket and underlying transport protocols
\end{itemize}

\subsection{Two Socket Types}

\begin{itemize}
    \item \textbf{UDP (User Datagram Protocol)}: Unreliable datagram service
    \item \textbf{TCP (Transmission Control Protocol)}: Reliable, byte stream-oriented service
\end{itemize}

\subsection{Application Example}

\begin{enumerate}
    \item Client reads a line of characters (data) from its keyboard and sends data to server
    \item Server receives the data and converts characters to uppercase
    \item Server sends modified data to client
    \item Client receives modified data and displays line on its screen
\end{enumerate}

\textcolor{blue}{[Summary: Socket programming enables communication between client and server applications. UDP provides connectionless, unreliable service while TCP provides connection-oriented, reliable service.]}

\textcolor{orange}{[Mnemonic: UDP = Unreliable Datagram Protocol; TCP = Trustworthy Connection Protocol]}

\section{Socket Programming with UDP}

\subsection{UDP Characteristics}

\begin{itemize}
    \item No "connection" between client and server
    \item No handshaking before sending data
    \item Sender explicitly attaches IP destination address and port number to each packet
    \item Receiver extracts sender IP address and port number from received packet
    \item Transmitted data may be lost or received out-of-order
    \item Provides \textit{unreliable} transfer of groups of bytes ("datagrams")
\end{itemize}

\subsection{Client/Server Socket Interaction: UDP}

\begin{figure}[h]
    \centering
    \begin{tabular}{|c|c|}
        \hline
        \textbf{Server}                             & \textbf{Client}                             \\
        \hline
        create socket, port=x:                      & create socket:                              \\
        serverSocket = socket(AF\_INET,SOCK\_DGRAM) & clientSocket = socket(AF\_INET,SOCK\_DGRAM) \\
        \hline
        read datagram from serverSocket             & Create datagram with serverIP address       \\
                                                    & and port=x; send datagram via clientSocket  \\
        \hline
        write reply to serverSocket                 & read datagram from clientSocket             \\
        specifying client address, port number      &                                             \\
        \hline
                                                    & close clientSocket                          \\
        \hline
    \end{tabular}
    \caption{UDP Client/Server Interaction Sequence}
\end{figure}

\subsection{Example Application: UDP Client}

\begin{minted}{python}
from socket import *
serverName = 'hostname'
serverPort = 12000
clientSocket = socket(AF_INET, SOCK_DGRAM)
message = raw_input('Input lowercase sentence:')
clientSocket.sendto(message.encode(), (serverName, serverPort))
modifiedMessage, serverAddress = clientSocket.recvfrom(2048)
print modifiedMessage.decode()
clientSocket.close()
\end{minted}

\subsection{Example Application: UDP Server}

\begin{minted}{python}
from socket import *
serverPort = 12000
serverSocket = socket(AF_INET, SOCK_DGRAM)
serverSocket.bind(("", serverPort))
print "The server is ready to receive"
while True:
    message, clientAddress = serverSocket.recvfrom(2048)
    modifiedMessage = message.decode().upper()
    serverSocket.sendto(modifiedMessage.encode(), clientAddress)
\end{minted}

\textcolor{blue}{[Summary: UDP socket programming involves connectionless communication where each datagram is sent independently. The server binds to a port and waits for incoming datagrams, while the client sends datagrams to the server's address and port.]}

\textcolor{teal}{[Concept Map: UDP → Connectionless → No handshaking → Explicit addressing → Unreliable → May lose/reorder data → Suitable for real-time applications]}

\section{Socket Programming with TCP}

\subsection{TCP Characteristics}

\begin{itemize}
    \item Client must contact server first
    \item Server process must be running first
    \item Server must have created socket (welcoming socket) for client contact
    \item When client creates socket: client TCP establishes connection to server TCP
    \item When contacted by client, server TCP creates new socket for that particular client
    \item Allows server to communicate with multiple clients simultaneously
    \item Source port numbers used to distinguish clients
    \item Provides reliable, in-order byte-stream transfer ("pipe")
\end{itemize}

\subsection{Client/Server Socket Interaction: TCP}

\begin{figure}[h]
    \centering
    \begin{tabular}{|c|c|}
        \hline
        \textbf{Server}                              & \textbf{Client}                          \\
        \hline
        create socket, port=x, for incoming request: &                                          \\
        serverSocket = socket()                      &                                          \\
        \hline
        wait for incoming connection request         & create socket, connect to hostid, port=x \\
        connectionSocket = serverSocket.accept()     & clientSocket = socket()                  \\
        \hline
        read request from connectionSocket           & send request using clientSocket          \\
        \hline
        write reply to connectionSocket              & read reply from clientSocket             \\
        \hline
        close connectionSocket                       & close clientSocket                       \\
        \hline
    \end{tabular}
    \caption{TCP Client/Server Interaction Sequence}
\end{figure}

\subsection{Example Application: TCP Client}

\begin{minted}{python}
from socket import *
serverName = 'servername'
serverPort = 12000
clientSocket = socket(AF_INET, SOCK_STREAM)
clientSocket.connect((serverName, serverPort))
sentence = raw_input('Input lowercase sentence:')
clientSocket.send(sentence.encode())
modifiedSentence = clientSocket.recv(1024)
print 'From Server:', modifiedSentence.decode()
clientSocket.close()
\end{minted}

\subsection{Example Application: TCP Server}

\begin{minted}{python}
from socket import *
serverPort = 12000
serverSocket = socket(AF_INET, SOCK_STREAM)
serverSocket.bind(("", serverPort))
serverSocket.listen(1)
print 'The server is ready to receive'
while True:
    connectionSocket, addr = serverSocket.accept()
    sentence = connectionSocket.recv(1024).decode()
    capitalizedSentence = sentence.upper()
    connectionSocket.send(capitalizedSentence.encode())
    connectionSocket.close()
\end{minted}

\textcolor{blue}{[Summary: TCP socket programming involves connection-oriented communication with a three-way handshake. The server creates a welcoming socket, accepts client connections, and creates dedicated sockets for each client, providing reliable, in-order data transfer.]}

\section{Chapter 2: Summary}

\subsection{Key Concepts Covered}

\begin{itemize}
    \item \textbf{Application Architectures}:
          \begin{itemize}
              \item Client-server architecture
              \item P2P (Peer-to-Peer) architecture
          \end{itemize}

    \item \textbf{Service Requirements Specification}:
          \begin{itemize}
              \item Reliability requirements
              \item Bandwidth requirements
              \item Delay requirements
          \end{itemize}

    \item \textbf{Internet Transport Service Model}:
          \begin{itemize}
              \item Connection-oriented, reliable: TCP
              \item Unreliable, datagrams: UDP
          \end{itemize}

    \item \textbf{Specific Protocols Studied}:
          \begin{itemize}
              \item HTTP (HyperText Transfer Protocol)
              \item SMTP (Simple Mail Transfer Protocol), IMAP (Internet Message Access Protocol)
              \item DNS (Domain Name System)
              \item P2P: BitTorrent
          \end{itemize}

    \item \textbf{Advanced Topics}:
          \begin{itemize}
              \item Video streaming techniques
              \item Content Delivery Networks (CDNs)
              \item Socket programming with TCP and UDP
          \end{itemize}
\end{itemize}

\subsection{Protocol Design Principles}

\begin{itemize}
    \item \textbf{Typical request/reply message exchange}:
          \begin{itemize}
              \item Client requests information or service
              \item Server responds with data, status code
          \end{itemize}

    \item \textbf{Message formats}:
          \begin{itemize}
              \item \textit{Headers}: Fields giving information about data
              \item \textit{Data}: Information (payload) being communicated
          \end{itemize}

    \item \textbf{Important design themes}:
          \begin{itemize}
              \item Centralized vs. decentralized architectures
              \item Stateless vs. stateful protocols
              \item Scalability considerations
              \item Reliable vs. unreliable message transfer
              \item "Complexity at network edge" principle
          \end{itemize}
\end{itemize}

\textcolor{blue}{[Summary: The application layer encompasses diverse protocols and architectures for network applications. Key considerations include reliability, performance requirements, and the choice between connection-oriented (TCP) and connectionless (UDP) transport services.]}

\textcolor{teal}{[Concept Map: Application Layer → Architectures (Client-Server, P2P) → Transport Services (TCP, UDP) → Protocols (HTTP, SMTP, DNS) → Advanced Topics (Streaming, CDNs) → Socket Programming]}

\section*{Exam Preparation}

\subsection*{Exam Questions}

\textcolor{red}{[Exam Questions:}
\begin{enumerate}
    \item Compare and contrast TCP and UDP socket programming. When would you choose one over the other?
    \item Explain the difference between a welcoming socket and a connection socket in TCP server programming.
    \item What are the key differences in how UDP and TCP handle addressing in socket programming?
    \item Write a Python code snippet for a UDP client that sends a message to a server and prints the response.
    \item Describe the sequence of socket method calls in a typical TCP client-server application.
    \item What are the advantages and disadvantages of connection-oriented vs. connectionless protocols?
    \item How does a TCP server handle multiple concurrent clients?
\end{enumerate}
\textcolor{red}{]}

\subsection*{Key Differences: TCP vs UDP}

\begin{table}[h]
    \centering
    \begin{tabular}{|l|l|l|}
        \hline
        \textbf{Feature} & \textbf{TCP}              & \textbf{UDP}           \\
        \hline
        Connection       & Connection-oriented       & Connectionless         \\
        Reliability      & Reliable                  & Unreliable             \\
        Ordering         & In-order delivery         & No ordering guarantees \\
        Handshaking      & 3-way handshake required  & No handshaking         \\
        Overhead         & Higher                    & Lower                  \\
        Use Cases        & Web, email, file transfer & DNS, streaming, VoIP   \\
        \hline
    \end{tabular}
    \caption{Comparison of TCP and UDP Protocols}
\end{table}

\subsection*{Study Tips}

\begin{itemize}
    \item Practice writing both UDP and TCP client-server code
    \item Understand the socket API method sequence for both protocols
    \item Memorize the key differences between connection-oriented and connectionless protocols
    \item Be able to explain when to use TCP vs UDP for different application requirements
    \item Review the protocol headers and message formats for HTTP, SMTP, and DNS
\end{itemize}

\end{document}