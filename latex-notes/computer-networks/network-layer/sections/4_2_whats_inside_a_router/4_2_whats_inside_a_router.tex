\section{4.2 What's Inside a Router?}

\subsection{Concept Overview: Unboxing the Data Plane}

\begin{enumerate}
    \item Primary function to transfer packets from incoming links to appropriate outgoing links
    \item Task split smong four interconnected componenets
          \begin{enumerate}
              \item Input Ports (Hardware):
                    \begin{enumerate}
                        \item Perform physical, link, and data-plane lookup for incoming packets
                        \item Is entry point for packets into router
                        \item Physically terminates incoming link, performs link-layer tasks to "unwrap" the packet, and perform lookup function to determine packet output port using forwarding table
                        \item Control packets are passed "up" to routing processor
                    \end{enumerate}
              \item Switching Fabric (Hardware):
                    \begin{enumerate}
                        \item "Heart" of router
                        \item Purpose to connect input ports to output ports
                    \end{enumerate}
                    Internal mechanism that moves packets from input ports to output ports
              \item Output Ports (Hardware):
                    \begin{enumerate}
                        \item Exit point for packets leaving router
                        \item Receives packets from switching fabric, queues them, performs necessary link-layer and physical-layer functions to "wrap" packet into new frame and finally transmits frame onto outgoing link
                    \end{enumerate}
                    Store, queue, and transmit packets received from switching fabric onto outgoing link
              \item Routing Processor (Software):
                    \begin{enumerate}
                        \item "Brain" of router
                        \item Primary component of control plane.
                        \item Implemented in software and runs on traditional CPU
                        \item Is not directly involved with main forwarding path
                              \begin{enumerate}
                                  \item In traditional router, routing processor executes routing protocols, maintains routing tables, and uses info to compute and update forwarding table
                                  \item In SDN router, routing processor is responsible for communicating with remote SDN controller, receiving forwarding table entries from it, and installing entriies into input ports
                              \end{enumerate}

                    \end{enumerate}
                    Executes control-plane functions like routing protocols and maintains/computers forwarding table
          \end{enumerate}
\end{enumerate}


\subsection{The Four Components of a Router}
\begin{verbatim}
                                  +---------------------+
                                  |  Routing Processor  |  (Control Plane)
                                  | (Software)          |
                                  +---------------------+
                                            |
                                            | (e.g., PCI Bus)
                                            |
+------------+     +------------+     +------------+     +------------+
| Input Port |     | Input Port |     | Output Port|     | Output Port|
| (Hardware) |     | (Hardware) |     | (Hardware) |     | (Hardware) |
+------------+     +------------+     +------------+     +------------+
       |                  |                  ^                  ^
       +------------------+------------------+------------------+
                          |   Switching Fabric |
                          |     (Hardware)     |
                          +--------------------+
\end{verbatim}

\subsection{Data Plane (Hardware) vs Control Plane (Software)}

Essential to understand why such components are built the way they are
\begin{itemize}
    \item Data Plane
          \begin{enumerate}
              \item Reason: Speed, executes in nanosecond timescale
              \item Example:
                    \begin{enumerate}
                        \item Consider 100 Gbps input link
                        \item New packet can arrive every few nanoseconds
                        \item For a min-sized 64-byte (512 bit) IP datagram, router only has 5.12 nanoseconds (512 bits / 100 Gbps) to process before next one arrives
                    \end{enumerate}
              \item Per-packet forwarding action far too fast for software-based implementation
          \end{enumerate}
    \item Control Plane
          \begin{enumerate}
              \item Reason: Complexity and Timescale, control-plane functions (like new route using Dijkstra's algo or commuinicating with SDN controller) are complex and happen on much slower timescale - milliseconds or even seconds
              \item Well suited for standard CPU running a software program.
          \end{enumerate}

\end{itemize}


\subsection{Roundabout Analogy}

\begin{itemize}
    \item Input Port: Entry road, which includes an attendant (lookup function)
    \item Switching Fabric: roundabout itself
    \item Output Port: exit ramp
    \item Destination-based Forwarding: Tell attendant find destination (i.e. Florida) and attendant looks it up, tells you "take 3rd exit", and you're on the way
    \item Generalized Forwarding: Attendants decision is more complex
          \begin{enumerate}
              \item "You're from NY? Take the slow road. You're from this state (license plate)? Take highway. Your car is not road-worthy? You're blocked"
          \end{enumerate}

    \item Bottlenecks: Analogy illustrates router's potential performance bottlenecks
          \begin{enumerate}
              \item Lookup bottleneck: Attendant is too slow. Cars pile up on entry road (input port)
              \item Switching bottleneck: Roundabout is too small or slow. Cars get stuck in middle (switch fabric)
              \item Output bottleneck: Too many cars want same exit ramp. Cars pil up on exit ramp (output port)
          \end{enumerate}
\end{itemize}





% TEMPLATE EXAMPLE
\subsection{Template Example}

\begin{itemize}
    \item X
          \begin{enumerate}
              \item Y
          \end{enumerate}
\end{itemize}
