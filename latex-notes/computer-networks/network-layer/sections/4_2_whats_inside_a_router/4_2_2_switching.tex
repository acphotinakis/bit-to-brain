\section{4.2.2 Switching Fabric}


\subsection{Three Switching Techniques}
Switching fabric is component that actually moves packets from input to output ports via three ways below

\begin{itemize}
    \item Switching via Memory
          \begin{enumerate}
              \item How it works:
                    \begin{enumerate}
                        \item Packets would arrive at input port, which would trigger an interrupt
                        \item CPU would copy packet into main system memory
                        \item CPU would perform lookup, find output port, and copy packet from memory to output port's buffer
                    \end{enumerate}
              \item Bottleneck
                    \begin{enumerate}
                        \item Routers total forwarding throughput is limited by memory bandwidth
                        \item Since packet must be written into memory and then read out, max throughput is B/2, where B is memory bus speed
                        \item Only one packet processed at a time
                    \end{enumerate}
              \item How it works (modern)
                    \begin{enumerate}
                        \item CPU no longer used
                        \item Instead, input port's hardware processor performs lookup and writes packet directory to output ports memory
                    \end{enumerate}
          \end{enumerate}
    \item Switching via a Bus
          \begin{enumerate}
              \item How it works:
                    \begin{enumerate}
                        \item All input ports and output ports hsare single, common bus
                        \item Input port receives a packet, performs lookup, and prepents a special "switch-internal label" to packet (like saying "this is for output 2"), and sents it onto the bus
                    \end{enumerate}
              \item All output ports receive packet, but only port that matches label will keep and transmit it
              \item Bottlenecks
                    \begin{enumerate}
                        \item Bus is shared resource, only one packet can cross bus at a time
                        \item Routers total throughput limited to bus speed
                        \item Fine for small enterprice or home router, but not for high-speed core routers
                    \end{enumerate}
          \end{enumerate}
    \item Switching via an Interconnection Network (Crossbar)
          \begin{enumerate}
              \item How it works:
                    \begin{enumerate}
                        \item This is highest-performance solution
                        \item Crossbar switch is a grid of 2N buses (N horizontal, N vertical) connecting N input ports to N output ports
                        \item A "crosspoint" (a transistor switch) exists at every intersection
                    \end{enumerate}
              \item When packet at input port A needs to go to output port Y, switch fabric controller closes crosspoint connecting A's horizontal bus to Y's vertical bus
              \item Key Property (parallelism)
                    \begin{enumerate}
                        \item A packet from input B can simutaneously be forwarded to output X by closing (B, X) crosspoint
                        \item Allows multiple packets to be transferred in parallel, as long as they are going to different output ports
                    \end{enumerate}
              \item Non-Blocking
                    \begin{enumerate}
                        \item Crossbar switch is non-blocking -- a packet detined for an idle output port is inever prevented from reaching it
                        \item However, if two packets from A and B both want to go to port Y, one will have to wait
                        \item This is output port contention, not a limitation of fabric itself
                    \end{enumerate}
          \end{enumerate}
\end{itemize}







% TEMPLATE EXAMPLE
\subsection{Template Example}

\begin{itemize}
    \item X
          \begin{enumerate}
              \item Y
          \end{enumerate}
\end{itemize}
