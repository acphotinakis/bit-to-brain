\section{4.2.4 Where Does Queuing Occur?}

\subsection{Overview}
This is a critical topic in router performance. Packet queues form when the packet arrival rate exceeds the forwarding rate. This can happen at both input and output ports. These queues are where packet delay is incurred and packet loss (when a queue overflows) occurs.

\begin{itemize}
    \item Output Queueing
          \begin{enumerate}
              \item Occurs when switching fabric is fast, but rate of packets arriving at output port (from multiple input ports) exceeds line speed of single output port
              \item Even a very fast switching fabric can feed packets to an output port faster than output link can transmit them, leading to a bottleneck at output port buffer
              \item Packet Loss and Buffer Management
                    \begin{enumerate}
                        \item If arrival rate to toutput queue persists, buffer will fill up
                              \begin{enumerate}
                                  \item Drop-Tail: default policy, when queue is full, newly arriving packet is dropped
                                  \item Active Queue Management (AQM)
                                        \begin{enumerate}
                                            \item More intelligent policy
                                            \item Proactively drop or mark (e.g. with ECN) packets before buffer is full
                                            \item Provides early congestion signal to senders (like TCP)
                                            \item Random Early Detection is classic example
                                        \end{enumerate}
                              \end{enumerate}
                    \end{enumerate}
          \end{enumerate}
    \item Input Queueing
          \begin{enumerate}
              \item Occurs when switching fabric is slower than combined speed of input ports (e.g. R_switch < N * R_line)
              \item Packets must wait in a queue at input port for their turn to cross fabric
              \item Head-of-the-Line Blocking
                    \begin{enumerate}
                        \item Specific problem in input-queued switches
                        \item Packet at head of an input queue is blocked from moving across fabric because desired output port is occupied by another packet (or because fabric is too slow)
                        \item Forces all subsequent packets in that input queue to wait, even if their desired output port is free
                        \item Can drastically limit throughput
                    \end{enumerate}
          \end{enumerate}
    \item Route Buffer Sizing: "How much buffering is 'enough'"?
          \begin{enumerate}
              \item Too little buffer = high packet loss, too much buffer = high queueing delay
              \item Rule of Thumb 1 (tranditional)
                    \begin{enumerate}
                        \item B = RTT * C, where B = buffer size, RTT = average round-trip time, C = link capacity
                        \item Example: 250ms RTT, 10 Gbps Link -> B = 0.25s * 20 Gbps = 2.5 Gbits of buffer, which is huge and expensive amount of memory
                        \item Purpose to fully utilize TCP conncetion pipeline without ricking premature packet drops
                    \end{enumerate}
              \item Rule of Thumb 2 (modern)
                    \begin{enumerate}
                        \item B = (RTT * C) / sqrt(N), where N = number of indepdent TCP flows
                        \item When large number of flows (N) are passing through a link, their traffic bursts tend to average out (statistical multiplexing)
                        \item Required buffer size to achieve good throughput and loss performance scales down with square root of N
                    \end{enumerate}
          \end{enumerate}

\end{itemize}




% TEMPLATE EXAMPLE
\subsection{Template Example}

\begin{itemize}
    \item X
          \begin{enumerate}
              \item Y
          \end{enumerate}
\end{itemize}
