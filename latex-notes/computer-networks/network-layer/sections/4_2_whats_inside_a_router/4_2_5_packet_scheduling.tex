\section{4.2.5 Packet Scheduling}


\subsection{Overview}

if an output port queue has mutliple packets waiting, packet schedulign discipline decides which packet to transmit next

\begin{itemize}
    \item FIFO
          \begin{enumerate}
              \item Simplest
              \item packets are transmitted in exact ordert they arrived
              \item FIFO provides no way to give priority to important packets (like VoIP) over less-important packets (like email)
          \end{enumerate}
    \item Priority Queueing
          \begin{enumerate}
              \item packets are classified into priority classes upon arrival, each class has its own queue
              \item Scheduler always transmits a packet from highest priority queue that has packets in it
              \item Packets within same priority class are served FIFO
              \item Preemption: typically non-preemptive, once a packet has started transmitting, it's not interrupted
              \item Problem: Starvation, high priority flow can completely block all low-priority flows
          \end{enumerate}
    \item Round Robin (RR) and Weighted Fair Queuing (WFQ)
          \begin{enumerate}
              \item "taking-turns" scheduler, packets are classified, schedular cycles through classes: send one packet from class 1, then one from class 2, then one from class 3, then back to class 1
              \item Work conserving, if a class's queue is empty, scheduler doesn't wait, it just skips to next class
              \item Weighted Fair Queueing
                    \begin{enumerate}
                        \item Each class i is assigned a weight, w_i
                        \item scheduler is still work conserving and round-robin
                        \item Service guarantee
                              \begin{enumerate}
                                  \item WFQ guarantees that each class i will receive a fractice of total bandwidth equal to: R * (w_i / \sigma(w_j)), where sum is over all classes that're currently active (have packets to send)
                              \end{enumerate}
                    \end{enumerate}
          \end{enumerate}

\end{itemize}



% TEMPLATE EXAMPLE
\subsection{Template Example}

\begin{itemize}
    \item X
          \begin{enumerate}
              \item Y
          \end{enumerate}
\end{itemize}
