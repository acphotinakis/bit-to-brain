\section{4.1.1 Forwarding and Routing: The Data and Control Planes}

Two key functions of the network layer, \textbf{forwarding} and \textbf{routing}, map directly to the \textbf{data plane} and \textbf{control plane}, respectively.

\subsection{1. Concept Overview: The Two Core Functions}

Welcome. In our last lecture, we completed our study of the transport layer, which provides a \emph{logical} communication service between \emph{processes} running on different hosts. Now, we move down the stack to examine the network layer, which provides the underlying \emph{host-to-host} communication service that the transport layer relies on.

The network layer's primary role seems simple: to move packets, called \textbf{datagrams}, from a sending host to a receiving host. However, to accomplish this, the network layer must perform two distinct and critical functions: \textbf{forwarding} and \textbf{routing}.

\begin{enumerate}
    \item Forwarding
          \begin{itemize}
              \item Is a \emph{local} action.
              \item The process of taking a packet that has arrived on one of a router's input links and moving it to the appropriate output link.
          \end{itemize}
    \item Routing (Control Plane Function)
          \begin{itemize}
              \item Is a \emph{network-wide} process that determines the end-to-end paths that packets take from a source host to a destination host.
          \end{itemize}
\end{enumerate}

\subsection{Forwarding: The "Data Plane"}

\begin{enumerate}
    \item Definition
          \begin{itemize}
              \item Forwarding refers to the router-local action of transferring a packet from an input link interface to the appropriate output link interface.
              \item Forwarding is the primary function of the network layer's \textbf{data plane}, which includes per-router operations performed on a packet as it moves through the router.
          \end{itemize}
    \item Mechanism
          \begin{itemize}
              \item A packet arrives at a router's input link.
              \item Router examines one or more fields in the packet's header.
              \item Uses header values to index into its forwarding table.
              \item Value found in the forwarding table entry indicates the router's output link interface to which the packet should be forwarded.
          \end{itemize}
    \item Timescale \& Implementation
          \begin{itemize}
              \item Forwarding is a fast (nanosecond) action that must be performed for every packet.
              \item Therefore, it is implemented in hardware.
          \end{itemize}
    \item Analogy
          \begin{itemize}
              \item Forwarding is like navigating a single highway interchange or roundabout: the driver makes a local, split-second decision based on signs.
          \end{itemize}
\end{enumerate}

\subsection{Routing (Control Plane Function)}

\begin{enumerate}
    \item Definition
          \begin{itemize}
              \item Routing refers to the network-wide process of calculating and determining the end-to-end paths that datagrams follow from source to destination.
              \item It is the primary function of the \textbf{control plane}, which governs network-wide routing logic.
          \end{itemize}
    \item Mechanism
          \begin{itemize}
              \item Routing is accomplished by routing algorithms that calculate paths for datagrams.
              \item Routing decisions are installed in the forwarding tables of routers.
          \end{itemize}
    \item Timescale \& Implementation
          \begin{itemize}
              \item Routing computations occur on much longer timescales (seconds to minutes).
              \item Typically implemented in software due to their complexity.
          \end{itemize}
    \item Analogy
          \begin{itemize}
              \item Routing is like planning an entire trip from Pennsylvania to Florida: the routing algorithm (driver) consults a map to select the best end-to-end path.
              \item Each router on the path forwards packets according to the plan.
          \end{itemize}
\end{enumerate}

\subsection{4.1.1.1 The Forwarding Table}

The routing (control plane) controls forwarding (data plane) via the \textbf{forwarding table}.

\begin{itemize}
    \item \textbf{Function:} Router examines header fields of incoming packets, indexes into the forwarding table, and forwards the packet according to the table entry.
    \item \textbf{Origin:} Entries are computed and installed by the control plane. Changes in topology or costs trigger updates.
\end{itemize}

\begin{verbatim}
                                 +---------------------+
                                 |  Routing Algorithm  | (Control Plane)
                                 +---------------------+
                                           |
                                           | (Computes & installs table)
                                           v
+-------------------+            +---------------------+
| Arriving Packet   |            | Local Forwarding    |
| (Header: 0110)    |  --------> | Table               | (Data Plane)
+-------------------+  (Match)   |---------------------|
                                 | 0100 | 3            |
                                 | 0110 | 2  <--------+ (Action: Forward to 2)
                                 | 0111 | 2            |
                                 | 1001 | 1            |
                                 +---------------------+
                                           |
                                           v
                                 (Packet forwarded to output link 2)
\end{verbatim}

\subsection{4.1.1.2 Approaches to the Control Plane}

\textbf{Approach 1: Traditional Per-Router Control}
\begin{itemize}
    \item Control plane runs in each router.
    \item Routing components communicate via routing protocols to compute forwarding tables.
    \item Distributed architecture.
\end{itemize}

\textbf{Approach 2: Software-Defined Networking (SDN)}
\begin{itemize}
    \item Control plane is physically separated from data plane.
    \item A remote controller computes forwarding tables and installs them on all routers.
    \item Routers act as pure data-plane devices.
    \item Controller runs in software, typically on a reliable data center, and communicates with routers using protocols like OpenFlow.
\end{itemize}

\subsection{4.1.1 Summary}

\begin{itemize}
    \item \textbf{Forwarding:} Local, fast, hardware-level data-plane function.
    \item \textbf{Routing:} Network-wide, slower, software-level control-plane function.
    \item \textbf{Data Plane:} Handles forwarding using the forwarding table.
    \item \textbf{Control Plane:} Computes paths and installs forwarding table entries.
    \item \textbf{Forwarding Table:} Links control plane and data plane.
    \item \textbf{Per-Router Control:} Traditional distributed routing model.
    \item \textbf{Logically Centralized Control (SDN):} Centralized controller computes forwarding tables.
\end{itemize}