\section{4.3.3 IPv6}

\subsection{Concept Overview}

IPv6 was developed primarily to address the impending exhaustion of the 32-bit IPv4 address space.
It solves the IPv4 address exhaustion problem by introducing 128-bit addressing and a simplified, fixed-length header.

\subsection{IPv6 Datagram Format}

\begin{itemize}
    \item \textbf{Key Changes from IPv4}
          \begin{enumerate}
              \item \textbf{Expanded Addressing (128 bits)}
                    \begin{enumerate}
                        \item Address size increased from 32 to 128 bits.
                        \item Provides $2^{128}$ possible addresses.
                    \end{enumerate}

              \item \textbf{Streamlined 40-byte Fixed Header}
                    \begin{enumerate}
                        \item IPv6 header is a fixed 40 bytes.
                        \item Simplifies router processing.
                    \end{enumerate}

              \item \textbf{Flow Labeling}
                    \begin{enumerate}
                        \item New field for identifying flows (e.g., a video stream).
                        \item Allows routers to provide special treatment (QoS/routing) to packets belonging to the same flow.
                    \end{enumerate}
          \end{enumerate}

    \item \textbf{IPv6 Header Fields}
          \begin{enumerate}
              \item \textbf{Version (4 bits):} Always set to 6.
              \item \textbf{Traffic Class (8 bits):} Like IPv4 TOS; supports priority/QoS.
              \item \textbf{Flow Label (20 bits):}
                    \begin{enumerate}
                        \item Labels packets belonging to a specific flow.
                        \item Routers can process these packets as a group.
                    \end{enumerate}
              \item \textbf{Payload Length (16 bits):} Length of data after the fixed 40-byte header.
              \item \textbf{Next Header (8 bits):}
                    \begin{enumerate}
                        \item Identifies the protocol of the payload (e.g., TCP, UDP).
                        \item Key to streamlined header structure.
                        \item If IPv6 options are included, this field points to an \textit{options header}.
                    \end{enumerate}
              \item \textbf{Hop Limit (8 bits):}
                    \begin{enumerate}
                        \item Same as IPv4 TTL.
                        \item Decrements at each router; packet dropped when it reaches 0.
                    \end{enumerate}
              \item \textbf{Source and Destination Addresses (128 bits each):}
                    \begin{enumerate}
                        \item Much larger address space.
                        \item Supports unicast, multicast, and anycast addressing.
                    \end{enumerate}
              \item \textbf{Data (Payload)}
          \end{enumerate}
\end{itemize}

\subsubsection{IPv6 Datagram Format (Structured View)}

\[
    \begin{array}{c|c|c}
        \text{Bits 0-3}                           & \text{Bits 4-11}     & \text{Bits 12-31} \\ \hline
        \text{Version (6)}                        & \text{Traffic Class} & \text{Flow Label} \\ \hline
        \multicolumn{2}{c}{\text{Payload Length}} & \text{Next Header}   & \text{Hop Limit}  \\ \hline
        \multicolumn{4}{c}{\text{Source Address (128 bits)}}                                 \\ \hline
        \multicolumn{4}{c}{\text{Destination Address (128 bits)}}                            \\ \hline
        \multicolumn{4}{c}{\text{Data (Payload)}}                                            \\
    \end{array}
\]

\subsection{Fields Dropped or Changed from IPv4}

\begin{itemize}
    \item \textbf{Fragmentation/Reassembly}
          \begin{itemize}
              \item Routers no longer fragment packets.
              \item If a packet is too large, router sends a \textit{Packet Too Big} ICMP message.
              \item Sender performs Path MTU discovery.
          \end{itemize}

    \item \textbf{Header Checksum}
          \begin{itemize}
              \item Removed—redundant due to link-layer and transport checksums.
              \item Removal speeds up per-hop processing (TTL/Hop Limit changes no longer require recalculating checksum).
          \end{itemize}

    \item \textbf{Options}
          \begin{itemize}
              \item Removed from base header.
              \item Now included via extension headers pointed to by the Next Header field.
          \end{itemize}
\end{itemize}

\subsection{Transitioning from IPv4 to IPv6}

\subsubsection{The Deployment Problem}

IPv4-only systems cannot handle IPv6 datagrams.
A global simultaneous “flag day” upgrade is impossible.

\subsubsection{Tunneling (Widely Used Transition Method)}

\begin{itemize}
    \item \textbf{Mechanism:} Encapsulate an IPv6 datagram inside an IPv4 datagram.
    \item \textbf{Process:}
          \begin{enumerate}
              \item Node B sends an IPv6 datagram to node E.
              \item If an IPv4 network lies in between, B encapsulates the IPv6 packet inside an IPv4 header.
              \item Outer IPv4 header uses IPv4 addresses of the tunnel endpoints (B and E).
              \item IPv4 routers forward normally, unaware of inner IPv6 datagram.
              \item Tunnel endpoint E removes IPv4 header and processes the IPv6 packet.
          \end{enumerate}
    \item \textbf{Purpose:} Allows IPv6 connectivity through existing IPv4 infrastructure.
\end{itemize}

\subsection{Key Terms and Definitions}

\begin{itemize}
    \item \textbf{Datagram:} Network-layer packet (IPv4 or IPv6).
    \item \textbf{Prefix (Network Portion):} High-order bits of an IP address, specified by the /x mask.
    \item \textbf{Subnet:} Devices sharing the same prefix and connected without a router.
    \item \textbf{CIDR:} Classless Interdomain Routing—current addressing system using variable-length prefixes.
    \item \textbf{DHCP:} Protocol for automatic assignment of IP addresses and configuration.
    \item \textbf{NAT:} Translates private addresses to a public address using ports.
    \item \textbf{Tunneling:} Encapsulation of one protocol inside another (e.g., IPv6 inside IPv4).
    \item \textbf{TTL / Hop Limit:} Prevents loops by decrementing at each hop.
    \item \textbf{IPv6:} Modern Internet Protocol using 128-bit addressing with simplified headers.
\end{itemize}

\subsection{Core Relationships}

\begin{itemize}
    \item \textbf{IP Address vs. Interface:} IP addresses are assigned to interfaces, not devices.
    \item \textbf{IP Addressing vs. Fragmentation:} IPv4 allows fragmentation; IPv6 does not (sender must handle MTU).
    \item \textbf{DHCP vs. NAT:}
          \begin{itemize}
              \item DHCP assigns local addresses.
              \item NAT maps private addresses to a public one for external communication.
          \end{itemize}
\end{itemize}

\subsection{Key Insights / Takeaways}

\begin{itemize}
    \item IPv4 uses a 20-byte header with fields that require per-hop processing (TTL, checksum).
    \item IP addressing is hierarchical to enable scalable routing via CIDR.
    \item NAT conserves addresses but breaks the end-to-end principle.
    \item IPv6 provides:
          \begin{itemize}
              \item vast address space (128 bits),
              \item simplified 40-byte header,
              \item no router fragmentation,
              \item flow label support,
              \item and relies on tunneling for transition from IPv4.
          \end{itemize}
\end{itemize}

% TEMPLATE EXAMPLE
\subsection{Template Example}
\begin{itemize}
    \item X
          \begin{enumerate}
              \item Y
          \end{enumerate}
\end{itemize}
