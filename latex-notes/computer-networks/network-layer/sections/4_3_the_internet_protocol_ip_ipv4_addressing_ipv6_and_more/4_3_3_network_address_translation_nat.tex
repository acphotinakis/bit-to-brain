
\section{4.3.3 Network Address Translation (NAT)}

\subsection{Concept Overview}

DHCP and CIDR helped, but IPv4 address space still faced exhaustion

\begin{itemize}
    \item NAT
          \begin{enumerate}
              \item Designed for SOHO (Small Office/Home Office) networks where many internal devices share single public IP address
          \end{enumerate}
    \item NAT-enabled Router
          \begin{enumerate}
              \item To the Internet: Entire home network looks like single device with a single IP address
              \item To the Home: Home network is a private network, using a private address space (e.g. 10.0.0.0/24)
                    \begin{enumerate}
                        \item These addresses are not routable on public internet and are reused by millions of homes
                    \end{enumerate}
          \end{enumerate}
\end{itemize}

\subsection{Technical Mechanism: NAT Translation Table}

How does NAT map one public address to many private addresses? Uses port numbers.

NAT router maintains NAT Translation table. This table mapes (Private IP, Private Port) pairs to (Public IP, Public Port) pairs

\begin{enumerate}
    \item Step-by-Step Operations
          \begin{enumerate}
              \item Outbound packet
                    \begin{enumerate}
                        \item Your laptop (10.0.0.1) sends packet to google.com (128.119.40.186)
                        \item Your OS picks a source port (e.g. 3345)
                        \item Packet arriving at NAT router: (Src IP: 10.0.0.1, Src Port: 3345), (Dest IP: 128.119.40.186, Dest Port: 80)
                        \item NAT Router
                              \begin{enumerate}
                                  \item Saves this mapping, then creates a new, unique public source port (e.g. 5001)
                                  \item Adds it to table: (10.0.0.1, 3345) <--> (138.76.29.7, 5001)
                              \end{enumerate}
                        \item Router sends modified packet to Internet. Web server (google.com) thinks request came from (138.76.29.7)
                        \item
                    \end{enumerate}
              \item Inbound Packet
                    \begin{enumerate}
                        \item google.come sends a replay, which is addressed to source of packet it received
                        \item Packet arriving at NAT router
                              \begin{enumerate}
                                  \item Src IP: 128.119.40.186, Src Port: 80
                                  \item Dest IP: 138.76.29.7, Dest Port: 5001
                              \end{enumerate}
                        \item NAT Router
                              \begin{enumerate}
                                  \item Looks up destination (138.76.29.7, 5001) in translation table.
                                  \item Finds matching internal address: (10.0.0.1, 3345).
                                  \item Rewrites the packet header again:
                                        \begin{enumerate}
                                            \item Src IP: 128.119.40.186, Src Port: 80
                                            \item Dest IP: 10.0.0.1, Dest Port: 3345
                                        \end{enumerate}
                                  \item The router forwards this packet into the home network, and it arrives at your laptop (10.0.0.1) on the correct port (3345).
                              \end{enumerate}
                    \end{enumerate}
          \end{enumerate}
\end{enumerate}


\subsection{Practical and Intuitive View: Controversy}

NAT is highly controversial among network purists

\begin{itemize}
    \item Pros
          \begin{enumerate}
              \item Saves IP Addresses: A single public IP can be shared by thousands of private IPs
              \item Security:
                    \begin{enumerate}
                        \item Internal devices are not directly addressable from outside world
                        \item Incoming packet is dropped unless there's an existing entry in translation table
                    \end{enumerate}
              \item Easy Management: Can add devices to home network without needing a new public IP from your ISP
          \end{enumerate}
    \item Cons
          \begin{enumerate}
              \item Breaks End-to-End Principle:
                    \begin{enumerate}
                        \item Port numbers are a Layer 4 (transport) concenpt meant to identify processes
                        \item NAT is Layer 3 (network) device that is reading and rewriting Layer 4 fields
                        \item Violation of layered architecture
                    \end{enumerate}
              \item Breaks P2P and Servers
                    \begin{enumerate}
                        \item External users can't connect to you since no public, routable address
                        \item Breaks P2P apps (like file sharing or games) and makes it harder to run a server from home
                        \item Requires special "NAT Traversal" techniques or manual "port forwarding" configurations
                    \end{enumerate}
              \item Middlebox
                    \begin{enumerate}
                        \item NAT is a "middlebox", a device that does more than just routing
                        \item Is complex, stateful device in middle of network, which violdates original internet design of simple core and smart edges
                    \end{enumerate}
          \end{enumerate}
\end{itemize}





% TEMPLATE EXAMPLE
\subsection{Template Example}

\begin{itemize}
    \item X
          \begin{enumerate}
              \item Y
          \end{enumerate}
\end{itemize}
