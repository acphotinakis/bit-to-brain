\section{4.3.2 IPv4 Addressing}


\subsection{Concept Overview: Interfaces and Subnets}
\begin{itemize}
    \item IP Addresses and Interfaces
          \begin{enumerate}
              \item IP addr is 32 bits long
              \item IP addr associated with a router or host interface, not device itself
              \item Router has multiple interfaces (one per link), and thus multiple IP addrs
              \item Hosts typically have one interface (and one IP address) connecting them to network
              \item Addresses are writting in dotted-decimal notation
          \end{enumerate}
\end{itemize}

\subsection{Technical Mechanism: The IP Subnet}

\begin{itemize}
    \item IP addresses not assigned randomly
    \item Portion of address is determined by network interface connected to
    \item /24 notation is subnet mask
          \begin{enumerate}
              \item Indicates that leftmost 24 bits define subnet, and remaining (32-24) = 8 bits identify specific interfaces on that subnet
          \end{enumerate}
    \item Recipe for Identifying Subnets
          \begin{enumerate}
              \item Detach every interface from its host or source
              \item This creates "islands" of isolated networks
              \item Each island is a subnet
          \end{enumerate}
    \item Subnets
          \begin{enumerate}
              \item IP subnet is a network that connects multiple host interfaces and router interfaces,
              \item Forms isolated network without intervening routers
          \end{enumerate}
    \item Subnet Addressing
          \begin{enumerate}
              \item Interfaces on given subnet share same high-order bits of their IP addresses
              \item Notation (CIDR)
                    \begin{enumerate}
                        \item A subnet address is denoted by form a.b.c.d/x, where /x indicates number of high-order bits that constitute network portion of address
                    \end{enumerate}
          \end{enumerate}

\end{itemize}


\subsection{Concept: CIDR (Classless Inter-Domain Routing)}

\begin{enumerate}
    \item Is a flexible IP address assignment strategy that replaced older class-based system
    \item Introduced /x notation, where x specifies how many bits of the address form network prefix
\end{enumerate}


\begin{itemize}
    \item Mechanism
          \begin{enumerate}
              \item CIDR generalizes traditional subnet and hsot division
                    \begin{enumerate}
                        \item In address a.b.c.d/x
                        \item x most significant bits are prefix
                        \item Remaining 32 - x bits are host portion
                    \end{enumerate}
              \item Allows networks to have variable-length prefixes, making address allocation more flexible
          \end{enumerate}
    \item Scalability via Address Aggregation
          \begin{enumerate}
              \item Organizations are assigned contiguous blocks of IP addresses with a common prefix
              \item Routers outside of org only need one entry in their forwarding tables.
                    \begin{enumerate}
                        \item "To reach any address startingt with 200.23.16.0/20, forward to that orgs ISP"
                    \end{enumerate}
              \item Ability to use single prefix to advertise many networks = address aggregation (or route aggregation)
              \item Essential for keeping global routing tables small and manageble
          \end{enumerate}
\end{itemize}

\subsection{Obtaining and Managing IP Addresses}
\begin{enumerate}
    \item Getting a block of addresses (for an ISP or large org)
          \begin{enumerate}
              \item Global authority is ICANN (Internet Corporation for Assigned Names and Numbers)
              \item ICANN allocates address blocks to Regional Internet Registries (RIRs)
              \item An ISP (like Comcast, or Verizon) gets its address blocks from its RIR
              \item An org gets its address blocks from its ISP
              \item Hierarchical allocation allows for route aggregation
          \end{enumerate}
    \item Getting a host address (DHCP)
          \begin{enumerate}
              \item Once org has blocks of addresses (e.g. 68.85.2.0/24) it needs to assign individual addresses to its hosts
              \item Done automatically by Dynamic Host Configuration Protocol (DHCP)
              \item DHCP - is a plug and play protocol. When your laptop connects to a network, uses DHCP to automatically get
                    \begin{enumerate}
                        \item Its IP address (e.g. 68.85.2.101)
                        \item Its subnet mask (e.g. /24)
                        \item IP address of default gateway (first hop router)
                        \item IP address of local DNS server
                    \end{enumerate}
              \item How DHCP works?
                    \begin{enumerate}
                        \item DHCP Discover
                              \begin{enumerate}
                                  \item New host client sends broadcast message (Dest IP: 255.255.255.255, Souce IP: 0.0.0.0)
                                  \item Asks "is there a DHCP server out there?"
                              \end{enumerate}
                        \item DHCP Offer
                              \begin{enumerate}
                                  \item A DHCP server (typically on the router) receives the discover and replies with offer message
                                  \item Includes proposing an IP address, lease time, etc.
                              \end{enumerate}
                        \item DHCP Request
                              \begin{enumerate}
                                  \item Client formally requests offered address by sending a broadcast "request" message
                                  \item This tells any other servers that sent offers that they weren't chosen
                              \end{enumerate}
                        \item DHCP ACK
                              \begin{enumerate}
                                  \item Server confirms allocation with a "DHCP ACK" message
                                  \item Client can now use IP address
                              \end{enumerate}
                    \end{enumerate}
          \end{enumerate}
\end{enumerate}



% TEMPLATE EXAMPLE
\subsection{Template Example}

\begin{itemize}
    \item X
          \begin{enumerate}
              \item Y
          \end{enumerate}
\end{itemize}
