\section{4.4 Generalized Forwarding and SDN}

\subsection{Concept Overview}

\begin{itemize}
    \item Traditional Model of Fowarding
          \begin{enumerate}
              \item Router makes forwarding decision based on single field in packet header, dest IP address
              \item Router sole job to perform lookup on dest IP address (using longest prefix matching) and send packet to correct output port
          \end{enumerate}
    \item Generalized Forwarding
          \begin{enumerate}
              \item Describes a "match-plus-action" paradigm where a packet switch (which could be a router or a switch) makes decision based on wide range of fields in packet header, spanning multiple protocol layers
              \item Match-plus-action abstraction
                    \begin{enumerate}
                        \item Instead of just matching dest IP, switch can match on any combination of fields, such as src IP, dest port, source MAC addresses, VLAN tag, etc.
                        \item Action:
                              \begin{enumerate}
                                  \item Instead of just forwarding, switch can perform variety of actions, such as
                                        \begin{enumerate}
                                            \item Forward packet to one or more output ports (e.g. unicast, multicast, or broadcast)
                                            \item Load balance packet across mutliple output ports
                                            \item Rewrite head values
                                            \item Block/drop packet
                                            \item Send packet to special server or controller for further processing (e.g. deep packet inspection)
                                        \end{enumerate}
                              \end{enumerate}
                    \end{enumerate}
              \item Flow Table
                    \begin{enumerate}
                        \item Generalizes simple forwarding table into a more complex flow table, flow table is "program" for packet switch's data plane
                    \end{enumerate}
              \item Connection to SDN
                    \begin{enumerate}
                        \item Generalized forwarding model is data-plane foundation of Software-Defined Networking
                        \item Because "match" and "action" rules can be complex and interdepedent, they're not computed by switch itself
                        \item Instead computed, installed, and updated by a remote, logically centralized controller
                    \end{enumerate}
          \end{enumerate}
\end{itemize}

\subsection{OpenFlow Flow Table}

\begin{itemize}
    \item match plus action table is called flow table. Each entry in table has three key components

          \begin{enumerate}
              \item Set of header field values (the "match"): Patterns used to match against incoming packet. Matching performed very fast, by TCAMs
              \item Set of counters (the "statistics"): Counters are updated as packets match rule. Crucial for network management and analystics. Coutners track number of packets matched by rule and time since rule was last updated
              \item Set of Actions (the "action"): Instructiosn to execute when a packet matches a rule. Can be a list of actions (e.g. "rewrite source port to 80, then forward to port 3")
          \end{enumerate}
    \item If packet arrives and matches no entry in flow table, can either be dropped, or more commonly, forward to remote controller for a decision. Controller can decide what to do and potentially install a new flow rule for this packet's flow
\end{itemize}





% TEMPLATE EXAMPLE
\subsection{Template Example}

\begin{itemize}
    \item X
          \begin{enumerate}
              \item Y
          \end{enumerate}
\end{itemize}
