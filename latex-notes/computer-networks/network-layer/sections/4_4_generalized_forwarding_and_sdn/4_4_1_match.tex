\section{4.4.1 Match}

\subsection{Concept Overview}

\begin{itemize}
    \item "Match" component of OpenFlow is what makes it powerful. Defines what parts of a packet switch can look at
\end{itemize}


\subsection{Technical Mechanism}

\begin{verbatim}
+--------------+------------------+------------------+---------+---------+ ...
| Ingress Port | Src MAC          | Dst MAC          | Eth Typ | VLAN ID | ...
+--------------+------------------+------------------+---------+---------+ ...
  (Switch Port) <---------- Link Layer (Layer 2) ------------------>

... +---------+---------+---------+---------+---------+-----------+-----------+
... | VLAN Pri| IP Src  | IP Dst  | IP Proto| IP TOS  | TCP/UDP Src| TCP/UDP Dst|
... +---------+---------+---------+---------+---------+-----------+-----------+
    (L2) <----------- Network Layer (Layer 3) ---------> <--- Transport (L4) -->
\end{verbatim}

\begin{itemize}
    \item OpenFlow 1.0 defines 11 header fields (plus ingress port) that can be matched
          \begin{enumerate}
              \item Ingress Port: physical port on which packet arrived at swtich
              \item Link-layer (L2) fields:
                    \begin{enumerate}
                        \item Src and Dest MAC Address: allows switch to operate as a Layer 2 switch
                        \item Ethernet Type: identifies protocol in payload
                        \item VLAN fields: allows for handling of Virtual LANs
                    \end{enumerate}
              \item Network-Layer (L3) Fields:
                    \begin{enumerate}
                        \item IP Src and Dest Address: allows switch to operate as a Layer 3 router
                        \item IP Protocol: Identifies transport protocol
                        \item IP TOS: allows matching on quality-of-service bits
                    \end{enumerate}
              \item Transport-Layer (L4) Fields
                    \begin{enumerate}
                        \item TCP/UDP Src and Dst Port: Allows for fine-grained, application aware rules (e.g. trading port 80 traffic differently from port 25 traffic)
                    \end{enumerate}
          \end{enumerate}
\end{itemize}

\subsection{Practical and Intuitive View: Wildcards and Priorities}

Made even more flexible by two additional features

\begin{itemize}
    \item Wildcards
          \begin{enumerate}
              \item Flow table entries can have wildcards for any field
              \item Example: a match entry with IP Dst = 128.119.*.* and TCP Dst Port = 80, would match all web traffic destined for any host on 128.119.0.0/16 subnet
          \end{enumerate}
    \item Priority
          \begin{enumerate}
              \item Packet might match multiple rules
              \item Example: one rule might match IP Dst = 128.119.*.*, while a second, more specific rule matches IP Dst = 128.119.40.1
              \item Flow table includes a priority field for each entry, and switch must apply action correspeonding to highest-priority matching rule
          \end{enumerate}
\end{itemize}




% TEMPLATE EXAMPLE
\subsection{Template Example}

\begin{itemize}
    \item X
          \begin{enumerate}
              \item Y
          \end{enumerate}
\end{itemize}
