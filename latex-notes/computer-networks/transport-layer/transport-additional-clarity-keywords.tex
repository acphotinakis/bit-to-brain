\documentclass[12pt]{article}
\usepackage[utf8]{inputenc}
\usepackage{amsmath, amssymb}
\usepackage{xcolor}
\usepackage{geometry}
\usepackage{hyperref}
\usepackage{fancyhdr}
\usepackage{enumitem}
\usepackage{minted}
\usepackage{booktabs}
\usepackage{graphicx}
\usepackage{tikz}
\usepackage{caption}
\usetikzlibrary{shapes, arrows, positioning}

% -------------------------------
% Page and link settings
% -------------------------------
\geometry{margin=1in}
\hypersetup{
    colorlinks=true,
    linkcolor=blue,
    urlcolor=cyan
}

% -------------------------------
% Custom Commands & Metadata
% -------------------------------
\newcommand{\TOPICTITLE}{Transport Layer - Additional Clarity Keywords}

\pagestyle{fancy}
\fancyhf{}
\fancyhead[L]{\textbf{\TOPICTITLE}}
\fancyhead[R]{\thepage}
\setlength{\headheight}{15pt}

\title{\TOPICTITLE\\\large Study-Ready Notes}
\author{Compiled by Andrew Photinakis}
\date{\today}

% -------------------------------
% Begin Document
% -------------------------------
\begin{document}

\maketitle
\tableofcontents
\newpage

\section{Transport Layer in Computer Networks}

\subsection{Core Definitions}
\begin{itemize}
    \item \textbf{Transport Layer:} Layer 4 of the OSI model responsible for end-to-end communication and data delivery between hosts.
    \item \textbf{Segment:} A unit of data encapsulated by the transport layer for transmission.
    \item \textbf{Port:} Logical endpoint identifying specific applications/services on a host.
    \item \textbf{Flow Control:} Mechanism to prevent sender from overwhelming the receiver.
    \item \textbf{Error Control:} Mechanism to detect and correct errors in transmitted data.
\end{itemize}

\subsection{Keyword Breakdown}
\begin{itemize}
    \item \textbf{TCP (Transmission Control Protocol):} Connection-oriented, reliable, ensures ordered delivery.
    \item \textbf{UDP (User Datagram Protocol):} Connectionless, faster, but unreliable and unordered.
    \item \textbf{Three-Way Handshake:} Process to establish TCP connection (SYN, SYN-ACK, ACK).
    \item \textbf{Checksum:} Field used for error detection in segments/datagrams.
    \item \textbf{Sliding Window:} Technique for flow control and efficient data transmission.
\end{itemize}

\subsection{Stepwise Mechanism}
\begin{enumerate}
    \item \textbf{TCP Connection Establishment:}
          \begin{itemize}
              \item Client sends SYN.
              \item Server responds with SYN-ACK.
              \item Client sends ACK; connection established.
          \end{itemize}
    \item \textbf{Data Transmission:}
          \begin{itemize}
              \item Segmenting large messages.
              \item Sending segments with sequence numbers.
              \item Receiver acknowledges received segments.
              \item Retransmit lost or corrupted segments.
          \end{itemize}
    \item \textbf{Connection Termination:}
          \begin{itemize}
              \item Four-way handshake: FIN, ACK, FIN, ACK.
          \end{itemize}
\end{enumerate}

\subsection{Examples \& Applications}
\begin{itemize}
    \item \textbf{Web Browsing:} HTTP/HTTPS over TCP.
    \item \textbf{Video Streaming:} UDP for live low-latency streams.
    \item \textbf{Email:} SMTP, IMAP, POP3 over TCP.
    \item \textbf{Gaming:} Real-time multiplayer using UDP.
\end{itemize}

\subsection{Comparisons / Contrasts}
\begin{itemize}
    \item \textbf{TCP vs UDP}
          \begin{itemize}
              \item TCP: Reliable, connection-oriented, slower.
              \item UDP: Unreliable, connectionless, faster.
          \end{itemize}
    \item \textbf{Flow Control vs Congestion Control}
          \begin{itemize}
              \item Flow Control: Manages sender vs receiver speed.
              \item Congestion Control: Manages network congestion to avoid packet loss.
          \end{itemize}
\end{itemize}

\subsection{Analogies}
\begin{itemize}
    \item Transport layer = postal service: ensures letters (data) reach the correct recipient (port) reliably (TCP) or quickly without guarantee (UDP).
\end{itemize}

\subsection{Visual / Diagram Description}
\begin{itemize}
    \item TCP Three-Way Handshake diagram:
          \begin{itemize}
              \item Client $\rightarrow$ SYN $\rightarrow$ Server
              \item Server $\rightarrow$ SYN-ACK $\rightarrow$ Client
              \item Client $\rightarrow$ ACK $\rightarrow$ Server
          \end{itemize}
    \item Optional figure: Segmentation and reassembly of a large message using sequence numbers.
\end{itemize}

\subsection{Concept Integration}
\begin{itemize}
    \item Interfaces with the Network Layer (IP) for addressing and routing.
    \item Provides reliable delivery for Application Layer protocols.
    \item Supports end-to-end communication across heterogeneous networks.
\end{itemize}

\subsection{Summary \& Study Aids}
\textcolor{blue}{[Summary: The transport layer ensures end-to-end data delivery, providing reliability, flow control, and error management, mainly through TCP and UDP protocols.]}

% \textcolor{orange}{[Mnemonic: \textbf{T.L. = TCP & UDP Learn}: Transport Layer manages \textbf{T}ransmission, \textbf{L}ogical ports, \textbf{U}dp, \textbf{D}elivery, \textbf{P}rotocols.]}



\end{document}