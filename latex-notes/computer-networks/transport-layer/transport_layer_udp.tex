\documentclass[12pt]{article}
\usepackage[utf8]{inputenc}
\usepackage{amsmath, amssymb}
\usepackage{xcolor}
\usepackage{geometry}
\usepackage{hyperref}
\usepackage{fancyhdr}
\usepackage{enumitem}
\usepackage{minted} % Code highlighting
\usepackage{booktabs} % Clean tables
\usepackage{tikz} % Optional for concept maps

\geometry{margin=1in}
\hypersetup{colorlinks=true, linkcolor=blue, urlcolor=cyan}

\pagestyle{fancy}
\fancyhf{}
\fancyhead[L]{\textbf{\TOPICTITLE}}
\fancyhead[R]{\thepage}

% -------------------------------
% Topic Metadata
% -------------------------------
\newcommand{\TOPICTITLE}{Transport Layer: UDP (User Datagram Protocol)}
\title{\TOPICTITLE\\\large Study-Ready Notes}
\author{Compiled by Andrew Photinakis}
\date{\today}

\setlength{\headheight}{15pt}

\begin{document}
\maketitle
\tableofcontents
\newpage

\section{Introduction to Transport Layer}
\begin{itemize}
    \item Transport-layer services provide end-to-end communication between applications
    \item Key functions include multiplexing and demultiplexing
    \item Two main transport protocols: UDP (connectionless) and TCP (connection-oriented)
    \item Additional topics: reliable data transfer, congestion control principles
\end{itemize}

\textcolor{blue}{[Summary: The transport layer enables application-to-application communication across networks, with UDP providing lightweight connectionless service and TCP offering reliable connection-oriented service.]}

\section{UDP: User Datagram Protocol}

\subsection{Basic Characteristics}
\begin{itemize}
    \item \textbf{"No frills," "bare bones"} Internet transport protocol
    \item \textbf{Best effort service} - segments may be:
          \begin{itemize}
              \item Lost during transmission
              \item Delivered out-of-order to application
          \end{itemize}
    \item \textbf{Connectionless} - no handshaking between UDP sender and receiver
    \item Each UDP segment handled independently of others
\end{itemize}

\subsection{Why Use UDP?}
\begin{itemize}
    \item \textbf{No connection establishment} - avoids RTT delay
    \item \textbf{Simple} - no connection state maintained at sender or receiver
    \item \textbf{Small header size} - minimal overhead
    \item \textbf{No congestion control} - can transmit as fast as desired
    \item \textbf{Robust} - can function even when network service is compromised
\end{itemize}

\textcolor{orange}{[Mnemonic: UDP = Uncomplicated, Direct, Prompt - emphasizes its simplicity and speed]}

\section{UDP Applications and Use Cases}

\subsection{Common UDP Applications}
\begin{itemize}
    \item \textbf{Streaming multimedia apps} - loss tolerant, rate sensitive
    \item \textbf{DNS (Domain Name System)} - quick name resolution
    \item \textbf{SNMP (Simple Network Management Protocol)} - network monitoring
    \item \textbf{HTTP/3} - modern web protocol built on UDP
\end{itemize}

\subsection{Reliability Over UDP}
\begin{itemize}
    \item If reliable transfer needed (e.g., HTTP/3):
          \begin{itemize}
              \item Add reliability mechanisms at application layer
              \item Implement congestion control at application layer
          \end{itemize}
    \item Applications can customize reliability to their specific needs
\end{itemize}

\section{UDP Protocol Specification [RFC 768]}

\subsection{Protocol Definition}
\begin{itemize}
    \item Standardized in RFC 768 (August 28, 1980)
    \item Provides datagram mode of packet-switched communication
    \item Assumes Internet Protocol (IP) as underlying protocol
    \item Transaction-oriented with minimal protocol mechanism
\end{itemize}

\subsection{Service Guarantees}
\begin{itemize}
    \item \textbf{No guaranteed delivery} - packets may be lost
    \item \textbf{No duplicate protection} - may receive copies
    \item \textbf{No ordered delivery} - out-of-order arrival possible
    \item Applications needing reliable streams should use TCP instead
\end{itemize}

\section{UDP Segment Format}

\subsection{Header Structure}
\begin{itemize}
    \item 32-bit (4-byte) header followed by data payload
    \item Header fields:
          \begin{itemize}
              \item \textbf{Source Port} (16 bits) - sender's port number
              \item \textbf{Destination Port} (16 bits) - receiver's port number
              \item \textbf{Length} (16 bits) - total segment size in bytes (header + data)
              \item \textbf{Checksum} (16 bits) - error detection field
          \end{itemize}
\end{itemize}

\subsection{Format Visualization}
\begin{verbatim}
    0               16              31
    +---------------+---------------+ 
    |  Source Port  | Destination Port|
    +---------------+---------------+ 
    |    Length     |   Checksum     |
    +---------------+---------------+ 
    |                                 |
    |          Data Payload          |
    |                                 |
    +-------------------------------+
\end{verbatim}

\textcolor{teal}{[Concept Map: UDP Segment → Header (Ports + Length + Checksum) + Data → IP Packet → Network Transmission]}

\section{UDP Transport Layer Actions}

\subsection{Sender Actions}
\begin{enumerate}
    \item Receives application-layer message
    \item Determines UDP segment header field values
    \item Creates UDP segment with header and data
    \item Passes segment to IP layer for transmission
\end{enumerate}

\subsection{Receiver Actions}
\begin{enumerate}
    \item Receives segment from IP layer
    \item Verifies UDP checksum header value
    \item Extracts application-layer message from payload
    \item Demultiplexes message to appropriate application via socket
\end{enumerate}

\subsection{SNMP Example}
\begin{itemize}
    \item SNMP client creates SNMP message
    \item UDP layer adds header and passes to IP
    \item SNMP server receives via IP, checks checksum, delivers to application
\end{itemize}

\section{UDP Checksum Mechanism}

\subsection{Goal and Purpose}
\begin{itemize}
    \item \textbf{Primary goal}: Detect errors (flipped bits) in transmitted segments
    \item Provides basic data integrity verification
    \item Covers UDP header, data, and pseudo-IP header information
\end{itemize}

\subsection{Sender Procedure}
\begin{enumerate}
    \item Treat UDP segment contents as sequence of 16-bit integers
    \item Include UDP header fields and IP addresses in calculation
    \item Compute one's complement sum of all 16-bit integers
    \item Place resulting checksum value in UDP checksum field
\end{enumerate}

\subsection{Receiver Procedure}
\begin{enumerate}
    \item Compute checksum of received segment using same method
    \item Compare computed checksum with received checksum field
    \item \textbf{Not equal}: Error detected - segment may be corrupted
    \item \textbf{Equal}: No error detected (but errors still possible due to checksum limitations)
\end{enumerate}

\section{Internet Checksum: Detailed Example}

\subsection{Calculation Process}
\begin{itemize}
    \item Add two 16-bit integers using one's complement arithmetic
    \item Example:
          \begin{itemize}
              \item First number: 1110 0110 0110 0110 (binary)
              \item Second number: 1101 0101 0101 0101 (binary)
              \item Sum: 1101 1101 1101 1011 (with carry)
          \end{itemize}
    \item \textbf{Wraparound}: Carry from most significant bit added back to result
\end{itemize}

\subsection{Mathematical Representation}
\[
    \begin{array}{cccccccccccccccc}
        1 & 1 & 1 & 0 & 0 & 1 & 1 & 0 & 0 & 1 & 1 & 0 & 0 & 1 & 1 & 0     \\
        1 & 1 & 0 & 1 & 0 & 1 & 0 & 1 & 0 & 1 & 0 & 1 & 0 & 1 & 0 & 1     \\
        \hline
        1 & 1 & 0 & 1 & 1 & 1 & 0 & 1 & 1 & 1 & 0 & 1 & 1 & 1 & 0 & 1 & 1 \\
    \end{array}
\]

\subsection{Checksum Weaknesses}
\begin{itemize}
    \item \textbf{Limited error detection capability}
    \item May not detect certain error patterns (e.g., swapped bytes)
    \item Example: Even when numbers change due to bit flips, checksum may remain unchanged
    \item Not cryptographically secure - easy to forge
\end{itemize}

\textcolor{blue}{[Summary: UDP checksum provides basic error detection using one's complement addition, but has limitations and cannot guarantee complete data integrity.]}

\section{Comprehensive UDP Summary}

\subsection{Protocol Characteristics}
\begin{itemize}
    \item \textbf{"No frills"} protocol with minimal overhead
    \item Segments may be lost or delivered out-of-order
    \item Best effort service: "send and hope for the best"
\end{itemize}

\subsection{UDP Advantages}
\begin{itemize}
    \item No setup/handshaking required (avoids RTT delay)
    \item Can function when network conditions are poor
    \item Provides basic error detection via checksum
    \item Enables application-specific reliability implementations
\end{itemize}

\subsection{Modern Applications}
\begin{itemize}
    \item Foundation for HTTP/3 and other modern protocols
    \item Allows building customized transport functionality
    \item Ideal for applications preferring timeliness over reliability
\end{itemize}

\textcolor{orange}{[Mnemonic: UDP Benefits = FAST - Fast setup, Application control, Simple, Timely delivery]}

\section{Study Questions}

\subsection{Conceptual Questions}
\begin{enumerate}
    \item Compare and contrast UDP and TCP in terms of reliability, connection establishment, and overhead.
    \item Explain why streaming applications often prefer UDP over TCP.
    \item Describe the UDP checksum calculation process and its limitations.
    \item What are the advantages of implementing reliability at the application layer rather than using TCP?
\end{enumerate}

\subsection{Technical Problems}
\begin{enumerate}
    \item Calculate the UDP checksum for a segment with source port 5200, destination port 53, length 32 bytes, and data "Hello".
    \item Explain what happens when a UDP receiver detects a checksum error.
    \item Design an application-layer protocol that adds reliability to UDP for file transfer.
\end{enumerate}

\subsection{Real-World Applications}
\begin{enumerate}
    \item Research and explain why HTTP/3 uses UDP instead of TCP.
    \item Identify three real-world applications that use UDP and explain why it's appropriate for each.
    \item Analyze the trade-offs between using raw UDP vs. building custom reliability vs. using TCP.
\end{enumerate}

\textcolor{teal}{[Concept Map: Transport Layer → UDP vs TCP → UDP: Connectionless + Best Effort + Applications (DNS, Streaming) + Checksum → Modern Uses (HTTP/3)]}



\end{document}