\documentclass[12pt]{article}
\usepackage[utf8]{inputenc}
\usepackage{amsmath, amssymb}
\usepackage{xcolor}
\usepackage{geometry}
\usepackage{hyperref}
\usepackage{fancyhdr}
\usepackage{enumitem}
\usepackage{minted} 
\usepackage{booktabs}
\usepackage{tikz}
\usetikzlibrary{shapes, arrows, positioning}

\geometry{margin=1in}
\hypersetup{colorlinks=true, linkcolor=blue, urlcolor=cyan}

\pagestyle{fancy}
\fancyhf{}
\fancyhead[L]{\textbf{\TOPICTITLE}}
\fancyhead[R]{\thepage}

% -------------------------------
% Topic Metadata
% -------------------------------
\newcommand{\TOPICTITLE}{4.1.1 Forwarding and Routing: The Data and Control Planes}
\title{\TOPICTITLE\\\large Study-Ready Notes}
\author{Compiled by Andrew Photinakis}
\date{\today}

\setlength{\headheight}{15pt}

\begin{document}
\maketitle
\tableofcontents
\newpage



\section{4.1.1 Forwarding and Routing: The Data and Control Planes}

Two key functions of network layer, forwarding and routing, map directly to data and control planes respectively.

\subsection{Forwarding}
\begin{enumerate}
    \item Definition
          \begin{itemize}
              \item Is the router-local action of transferring a packet from an input link interface to appropriate link interface within a single router.
          \end{itemize}
    \item Mechanism
          \begin{itemize}
              \item A packet arrives at a router's input link
              \item Router examines one or more fields in packet's header
              \item Uses header values to index into its forwarding table
              \item Value found in forwarding table entry indicates router's output link interface to which packet should be forwarded
          \end{itemize}
    \item Timescale \& Implementation
          \begin{itemize}
              \item Forwarding is fast (nanosecond) action that must be performed for every packet
              \item Therefore implemented in hardware
          \end{itemize}
    \item Analogy
          \begin{itemize}
              \item Forwarding is like a driver navigating a single highway interchange or roundabout
              \item Driver's decision (which exit ramp to take) is a local one, based on signs at that specific interchange and must be made quickly.
          \end{itemize}
\end{enumerate}


\subsection{Routing (Control Plane Function)}
\begin{enumerate}
    \item Definition
          \begin{itemize}
              \item Routing is network-wide process that determines end-to-end paths (or routes) that packets take from a soruce host to a destination host
          \end{itemize}
    \item Mechanism
          \begin{itemize}
              \item Routing is accomplished by routing algorithms
              \item Such algos calculate paths that datagrams will follow
          \end{itemize}
    \item Timescale \& Implementation
          \begin{itemize}
              \item Routing is complex calculation that hapepns on much longer timescale (seconds)
              \item Typically implemented in software
          \end{itemize}
    \item Analogy
          \begin{itemize}
              \item Routing is like planning the entire trip from PA to FL
              \item Before driving, you consult a map (routing algo) to choose best path
              \item Best path is sequence of road segments (links) connected by interchanges (routers)
          \end{itemize}
\end{enumerate}


\end{document}