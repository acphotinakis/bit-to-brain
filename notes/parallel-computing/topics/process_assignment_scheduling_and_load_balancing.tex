\documentclass[12pt]{article}
\usepackage[utf8]{inputenc}
\usepackage{amsmath, amssymb}
\usepackage{xcolor}
\usepackage{geometry}
\usepackage{hyperref}
\usepackage{fancyhdr}
\usepackage{enumitem}
\usepackage{minted} 
\usepackage{booktabs}
\usepackage{tikz} 
\usepackage{graphicx}

\geometry{margin=1in}
\hypersetup{colorlinks=true, linkcolor=blue, urlcolor=cyan}

\newcommand{\TOPICTITLE}{Process Assignment, Scheduling and Load Balancing}

\pagestyle{fancy}
\fancyhf{}
\fancyhead[L]{\textbf{\TOPICTITLE}}
\fancyhead[R]{\thepage}

\setlength{\headheight}{15pt}

\title{\TOPICTITLE\\\large Study-Ready Notes}
\author{Compiled by Andrew Photinakis}
\date{October 2nd, 2025}

\begin{document}
\maketitle
\tableofcontents
\newpage

\section{Introduction to Process Assignment and Scheduling}

\subsection{Basic Concepts}
\begin{itemize}
  \item \textbf{Process Assignment}: Mapping processes to processing elements (PEs)
        \begin{itemize}
          \item Considers: Process characteristics, Hardware/Software characteristics
          \item Answers the question: \textbf{Where?}
        \end{itemize}

  \item \textbf{Scheduling}: Determining when to start executing each task
        \begin{itemize}
          \item Types: Undirected, Directed
          \item Answers the question: \textbf{When?}
        \end{itemize}
\end{itemize}

[Summary] Process assignment determines where processes run, while scheduling determines when they execute. Both consider system and process characteristics for optimal performance.

\subsection{Programming Models}
\begin{itemize}
  \item \textbf{Definition}: Abstractions of CPU hardware, memory, and communication architectures
  \item \textbf{Types}: SPMD (Single Program Multiple Data), MPMD (Multiple Program Multiple Data), Shared Memory, Message Passing
  \item \textbf{Selection Criteria}:
        \begin{itemize}
          \item Application problem characteristics
          \item Available parallel computer architecture
          \item Knowledge of parallel algorithms and programming languages
        \end{itemize}
\end{itemize}

[Summary] Programming models provide abstractions for parallel computing, with selection depending on application needs and available hardware.

\section{Critical Factors in Parallel Computing}

\subsection{Granularity}
\begin{itemize}
  \item \textbf{Definition}: Ratio of computation to communication
  \item \textbf{Coarse vs Fine Granularity}:
        \begin{itemize}
          \item Coarse: Large computational work between communication events
          \item Fine: Frequent communication with smaller computation chunks
        \end{itemize}
  \item \textbf{Impact}: Higher computation/communication ratio $\rightarrow$ better speedup and efficiency
\end{itemize}

\subsection{Overheads}
\begin{itemize}
  \item \textbf{Types}:
        \begin{itemize}
          \item Synchronization costs
          \item Data communication costs
        \end{itemize}
  \item \textbf{Significance}: Overheads reduce effective parallel speedup
\end{itemize}

\subsection{Scalability}
\begin{itemize}
  \item \textbf{Definition}: Ability to maintain performance improvement with increasing processors
  \item \textbf{Influencing Factors}:
        \begin{itemize}
          \item Memory-CPU bandwidth
          \item Network communication capabilities
          \item Application algorithm design
          \item Programming language characteristics
          \item Process assignment strategy
        \end{itemize}
\end{itemize}

[Summary] Granularity, overheads, and scalability are critical factors affecting parallel performance, with granularity balancing computation and communication trade-offs.

\section{Processor and System Characteristics}

\subsection{Homogeneous vs Heterogeneous Systems}
\begin{itemize}
  \item \textbf{Homogeneous Processors}:
        \begin{itemize}
          \item All processors identical in type and capability
          \item Uniform computing and communication costs
        \end{itemize}

  \item \textbf{Heterogeneous Processors}:
        \begin{itemize}
          \item Processors with varying capabilities (speed, resources, software)
          \item Different computing and communication costs
          \item May require specific processors for certain processes
        \end{itemize}
\end{itemize}

\subsection{Network Characteristics}
\begin{itemize}
  \item \textbf{Homogeneous/Heterogeneous Networks}:
        \begin{itemize}
          \item Communication bandwidth may vary
          \item Considerations: Mobility, disconnection issues
        \end{itemize}
\end{itemize}

\subsection{Cost Modeling}
\begin{itemize}
  \item \textbf{Total Cost Formula}:
        \[
          \text{Total Cost} = \text{Computing Costs} + \text{Communication Costs}
        \]

  \item \textbf{Example Calculation}:
        \begin{itemize}
          \item Given process costs on different processors
          \item Communication cost between processors = 1 unit
          \item Assignment: (B,C,D) $\rightarrow$ P1; (A,E,F) $\rightarrow$ P2
          \item Total Cost = Sum of computation costs + communication costs
        \end{itemize}
\end{itemize}

[Summary] System heterogeneity affects assignment decisions, with total cost being the sum of computation and communication expenses.

\section{Problem and System Knowledge Requirements}

\subsection{Problem Analysis}
\begin{itemize}
  \item \textbf{Focus Areas}:
        \begin{itemize}
          \item Parallelize most time-consuming processes
          \item Avoid parallelizing trivial processes
          \item Identify bottlenecks: I/O, data dependencies
        \end{itemize}
\end{itemize}

\subsection{System Analysis}
\begin{itemize}
  \item \textbf{Processor Characteristics}:
        \begin{itemize}
          \item Speed, memory capacity
        \end{itemize}

  \item \textbf{Topology Types}:
        \begin{itemize}
          \item Mesh, Tree, Hypercube, 3-D Mesh
        \end{itemize}

  \item \textbf{Communication Patterns}:
        \begin{itemize}
          \item Processor-Processor
          \item Processor-Memory
          \item Memory-Memory
        \end{itemize}
\end{itemize}

\section{Decomposition Strategies}

\subsection{Domain Decomposition}
\begin{itemize}
  \item \textbf{Approach}: Divide data into discrete chunks
  \item \textbf{Applications}: Matrix operations, Image processing
  \item \textbf{Goal}: Maintain high $\frac{\text{cost of computing (R)}}{\text{cost of communication (C)}}$ ratio
  \item \textbf{Considerations}: Match system (R,C) with application (r,c)
\end{itemize}

\subsection{Functional Decomposition}
\begin{itemize}
  \item \textbf{Approach}: Assign different functions to different processors
  \item \textbf{Applications}: Signal processing (pipelined filter stages)
  \item \textbf{Key Principle}: Maintain matching R/C to r/c ratios for improved parallelism
\end{itemize}

[Summary] Domain decomposition partitions data, while functional decomposition partitions functions, both aiming to optimize computation-communication ratios.

\section{Process Assignment and Scheduling Types}

\subsection{Static Scheduling}
\begin{itemize}
  \item \textbf{Characteristics}:
        \begin{itemize}
          \item Problem and process complexities known \textit{a priori}
          \item Fixed assignments determined before execution
        \end{itemize}

  \item \textbf{Applications}: Traditional parallel systems with predictable workloads
\end{itemize}

\subsection{Dynamic Scheduling}
\begin{itemize}
  \item \textbf{Characteristics}:
        \begin{itemize}
          \item Processor availability changes over time
          \item Adapts to system state changes
          \item Combined with process migration and load balancing
        \end{itemize}

  \item \textbf{Applications}:
        \begin{itemize}
          \item Cloud systems (shared resources)
          \item Mobile systems (mobility, battery constraints)
        \end{itemize}
\end{itemize}

[Summary] Static scheduling works with known parameters, while dynamic scheduling adapts to changing system conditions.

\section{Process Scheduling Approaches}

\subsection{Centralized vs Decentralized}
\begin{itemize}
  \item \textbf{Centralized Scheduling}:
        \begin{itemize}
          \item Single controller makes all decisions
          \item Pros: Consistent, global view
          \item Cons: Single point of failure, scalability issues
        \end{itemize}

  \item \textbf{Decentralized Scheduling}:
        \begin{itemize}
          \item Distributed decision making
          \item Pros: Scalable, fault-tolerant
          \item Cons: Coordination overhead, potential inconsistencies
        \end{itemize}
\end{itemize}

\subsection{Primary Issues: Problem Perspective}
\begin{itemize}
  \item \textbf{Workload Distribution}:
        \begin{itemize}
          \item Distributing jobs and metadata
          \item Queue length management
        \end{itemize}

  \item \textbf{Session State Management}:
        \begin{itemize}
          \item Node stickiness (affinity)
          \item Cost of task reallocation
          \item Session state distribution
        \end{itemize}
\end{itemize}

\subsection{Primary Issues: System Perspective}
\begin{itemize}
  \item \textbf{Node Selection Policies}:
        \begin{itemize}
          \item Random selection
          \item Round Robin
          \item Shortest queue
          \item Threshold-based (queue length > threshold)
        \end{itemize}

  \item \textbf{Workload Metrics}:
        \begin{itemize}
          \item Queue Length
          \item CPU Utilization
          \item Response Time, Capacity, Network latency
          \item Probe limit
        \end{itemize}
\end{itemize}

\section{Optimal Scheduling Analysis}

\subsection{Mathematical Formulation}
\begin{itemize}
  \item \textbf{System Model}:
        \begin{itemize}
          \item N tasks: $p_1, p_2, p_3, \ldots, p_N$
          \item 2 processors: A and B
          \item Assignment: A: $p_1, p_2, \ldots, p_k$; B: $p_{k+1}, p_{k+2}, \ldots, p_N$
        \end{itemize}

  \item \textbf{Cost Components}:
        \begin{itemize}
          \item Computation cost for process $p_i = r_i$
          \item Communication cost between $p_i$ and $p_j = c_{i,j}$ if on different processors
          \item Communication cost = 0 if on same processor
        \end{itemize}
\end{itemize}

\subsection{Cost Analysis}
\begin{itemize}
  \item \textbf{Total Cost Formula}:
        \[
          \text{Total Cost} = r \times \max(k, N-k) + c \times [k \times (N-k)]
        \]

  \item \textbf{Special Cases}:
        \begin{itemize}
          \item All processes on one processor ($k = N$):
                \[
                  \text{Cost} = r \times N
                \]

          \item Equal division ($k = N/2$):
                \[
                  \text{Cost} = \frac{r \times N}{2} + c \times \left(\frac{N}{2} \times \frac{N}{2}\right) = \frac{1}{2}\left(r \times N + \frac{c \times N^2}{2}\right)
                \]
        \end{itemize}
\end{itemize}

\subsection{Parallelization Decision}
\begin{itemize}
  \item \textbf{Condition for Parallelization}:
        \[
          r \times N > \frac{1}{2}\left(r \times N + \frac{c \times N^2}{2}\right)
        \]
        Simplifies to:
        \[
          \frac{r}{c} > \frac{N}{2}
        \]

  \item \textbf{Decision Rule}:
        \begin{itemize}
          \item Equal distribution if $\frac{r}{c} > \frac{N}{2}$
          \item Use single processor if $\frac{r}{c} \leq \frac{N}{2}$
          \item Decision independent of number of processors
        \end{itemize}
\end{itemize}

\subsection{Assumptions and Limitations}
\begin{itemize}
  \item \textbf{Key Assumptions}:
        \begin{itemize}
          \item Total communication among all processes
          \item No overlap between computation and communication
          \item All processes communicate with each other
        \end{itemize}

  \item \textbf{Real-world Considerations}:
        \begin{itemize}
          \item Subset of processes communicate
          \item Computation and communication can often overlap
        \end{itemize}
\end{itemize}

[Summary] Optimal scheduling depends on the computation-to-communication cost ratio, with parallelization beneficial only when $r/c > N/2$.

\section{Clustering in Process Scheduling}

\subsection{Clustering Concept}
\begin{itemize}
  \item \textbf{Objective}: Group processes to minimize communication costs
  \item \textbf{Computation Cost}: $\sum_{i=1}^n r_i$
  \item \textbf{Communication Cost}: $\sum c_{in} + \sum c_{out}$
\end{itemize}

\subsection{Clustering Strategies}
\begin{itemize}
  \item \textbf{Input/Output Focus}:
        \begin{itemize}
          \item Consider incoming and outgoing communication costs
          \item Balance computation and communication within clusters
        \end{itemize}
\end{itemize}

\section{Dynamic Load Balancing}

\subsection{Basic Concepts}
\begin{itemize}
  \item \textbf{Definition}: Equitable distribution of load among processors
  \item \textbf{Goal}: Minimize difference between most heavily and lightly loaded processors
  \item \textbf{Key Principle}: Adjust based on monitored system state
\end{itemize}

\subsection{Dynamic Scheduling Algorithms}
\begin{itemize}
  \item \textbf{Sender-Initiated Algorithms}:
        \begin{itemize}
          \item Transfer policy: When queue length exceeds threshold
          \item Selection policy: Which process to transfer
          \item Location policy: Cost and distance considerations
        \end{itemize}

  \item \textbf{Receiver-Initiated Algorithms}:
        \begin{itemize}
          \item Triggered when queue length falls below threshold
        \end{itemize}
\end{itemize}

\subsection{Dynamic System Challenges}
\begin{itemize}
  \item \textbf{Sources of Dynamism}:
        \begin{itemize}
          \item Uncertainty in task execution times
          \item Dynamic task arrival and departure
          \item Changing processor availability
          \item Network condition variations
          \item Task priority changes
          \item Processor and network faults
        \end{itemize}

  \item \textbf{Exacerbating Factors}:
        \begin{itemize}
          \item Cloud systems (resource sharing)
          \item Mobile systems (mobility, energy constraints)
        \end{itemize}
\end{itemize}

[Summary] Dynamic load balancing adapts to changing system conditions using sender or receiver initiated approaches to maintain equitable load distribution.

\section{Load Balancing Methods and Challenges}

\subsection{Approaches to Load Balancing}
\begin{itemize}
  \item \textbf{Centralized vs Decentralized}:
        \begin{itemize}
          \item Centralized: Single controller, less scalable
          \item Decentralized: Distributed control, more scalable
        \end{itemize}

  \item \textbf{Information Collection}:
        \begin{itemize}
          \item How to obtain processor state information
          \item Centralized vs decentralized collection
          \item Periodic vs event-driven updates
          \item Threshold-based monitoring
        \end{itemize}
\end{itemize}

\subsection{Key Questions in Load Balancing}
\begin{itemize}
  \item \textbf{Transfer Policy}: Whether to move processes
  \item \textbf{Location Policy}: Where to move processes
  \item \textbf{Process Selection}: Which processes to move
  \item \textbf{Decision Architecture}: Centralized or distributed control
\end{itemize}

\subsection{Cost Considerations}
\begin{itemize}
  \item \textbf{Processing Overhead}:
        \begin{itemize}
          \item Data collection for load monitoring
          \item Decision making computations
        \end{itemize}

  \item \textbf{Network Overhead}:
        \begin{itemize}
          \item Distribution of load information
          \item Process migration costs
          \item Job redistribution
        \end{itemize}
\end{itemize}

\section{Static vs Dynamic Load Balancing}

\subsection{Static Load Sharing}
\begin{itemize}
  \item \textbf{Characteristics}:
        \begin{itemize}
          \item Fixed policy, no adaptation to system state
          \item Simple, low cost implementation
          \item Handles session state easily
          \item Cannot adjust to dynamic changes
        \end{itemize}
\end{itemize}

\subsection{Dynamic Load Balancing}
\begin{itemize}
  \item \textbf{Advantages}:
        \begin{itemize}
          \item Adapts to changing system conditions
          \item Better resource utilization
          \item Handles unpredictable workloads
        \end{itemize}

  \item \textbf{Disadvantages}:
        \begin{itemize}
          \item Complex implementation
          \item Significant overhead costs
          \item Session state management challenges
        \end{itemize}
\end{itemize}

\section{Process Migration}

\subsection{Migration Process}
\begin{itemize}
  \item \textbf{Steps}:
        \begin{itemize}
          \item Migration request initiation
          \item Process suspension on source host
          \item State transfer to destination host
          \item Process resumption on destination
          \item File server coordination
        \end{itemize}
\end{itemize}

\subsection{Migration Scenarios}
\begin{itemize}
  \item \textbf{Work Stealing}:
        \begin{itemize}
          \item Idle processor seeks work from busy ones
          \item "No more work $\rightarrow$ Find work elsewhere"
        \end{itemize}

  \item \textbf{Load Distribution}:
        \begin{itemize}
          \item Balance load across multiple processors
          \item Handle processor saturation scenarios
        \end{itemize}
\end{itemize}

\section{Distributed Scheduling Framework}

\subsection{System Components}
\begin{itemize}
  \item \textbf{Load Information Management}:
        \begin{itemize}
          \item Collects local node load information
          \item Disseminates information to other nodes
        \end{itemize}

  \item \textbf{Distributed Scheduling}:
        \begin{itemize}
          \item Makes migration decisions
          \item Determines when, where, and which processes to migrate
        \end{itemize}

  \item \textbf{Migration Mechanism}:
        \begin{itemize}
          \item Executes the actual process transfer
        \end{itemize}
\end{itemize}

\subsection{Implementation Considerations}
\begin{itemize}
  \item \textbf{Local Information Collection}: Monitoring node status
  \item \textbf{Information Dissemination}: Sharing load data
  \item \textbf{Migration Directives}: Decision rules for process movement
\end{itemize}

\section*{Study Aids}

\subsection*{Key Formulas}
\begin{itemize}
  \item Total Cost = Computing Costs + Communication Costs
  \item Optimal scheduling condition: $\frac{r}{c} > \frac{N}{2}$
  \item Clustering costs: Computation = $\sum r_i$, Communication = $\sum c_{in} + \sum c_{out}$
\end{itemize}

\subsection*{Important Concepts}
\begin{itemize}
  \item Granularity: Computation-to-communication ratio
  \item Homogeneous vs Heterogeneous systems
  \item Static vs Dynamic scheduling
  \item Centralized vs Decentralized control
  \item Sender vs Receiver initiated load balancing
\end{itemize}

\subsection*{Exam Questions}
\begin{enumerate}
  \item Compare and contrast static and dynamic scheduling approaches, including their advantages and limitations.

  \item Derive the condition for optimal parallelization in a 2-processor system and explain its practical implications.

  \item Describe three node selection policies for load balancing and discuss situations where each would be most appropriate.

  \item Explain the process migration mechanism and discuss the costs involved in dynamic load balancing.

  \item How does system heterogeneity affect process assignment decisions? Provide examples.
\end{enumerate}

[Mnemonic] "GCD SL" - Granularity, Cost, Dynamism for Scheduling and Load balancing

  [Concept Map]
\begin{itemize}
  \item Process Assignment $\rightarrow$ Where (Mapping)
  \item Scheduling $\rightarrow$ When (Timing)
  \item Load Balancing $\rightarrow$ How (Distribution)
  \item Static vs Dynamic approaches
  \item Centralized vs Decentralized control
  \item Homogeneous vs Heterogeneous systems
\end{itemize}

\end{document}