\documentclass[12pt]{article}
\usepackage[utf8]{inputenc}
\usepackage{amsmath, amssymb}
\usepackage{xcolor}
\usepackage{geometry}
\usepackage{hyperref}
\usepackage{fancyhdr}
\usepackage{enumitem}
\usepackage{minted} % Code highlighting
\usepackage{booktabs} % Clean tables}
\usepackage{tikz} % Optional for concept maps

\geometry{margin=1in}
\hypersetup{colorlinks=true, linkcolor=blue, urlcolor=cyan}

\newcommand{\TOPICTITLE}{Message Passing Interface (MPI)}

\pagestyle{fancy}
\fancyhf{}
\fancyhead[L]{\textbf\TOPICTITLE}
\fancyhead[R]{\thepage}

\title{\TOPICTITLE\\\large Study-Ready Notes}
\author{Compiled by Andrew Photinakis}
\date{}

\setlength{\headheight}{15pt}

\begin{document}
\maketitle
\tableofcontents
\newpage

% This LaTeX file should be saved at: parallel_computing/week04/message_passing_interface_mpi.tex

\section{Introduction to MPI}

\subsection{What is MPI?}
\begin{itemize}
    \item Message Passing Interface (MPI) - a specification, not a library
    \item Message-passing parallel programming model: data moved from address space of one process to another through cooperative operations
    \item Well-known implementations: Open MPI and MVAPICH2
    \item All parallelism is explicit: programmer responsible for identifying parallelism and implementing parallel algorithms
\end{itemize}

\textcolor{blue}{[Summary: MPI is a standardized message-passing specification for parallel programming where processes communicate by explicitly sending and receiving messages between their separate address spaces.]}

\subsection{Parallel Computing Systems}
\begin{itemize}
    \item Distributed Memory: Computers in a network
    \item Distributed Shared Memory: Computers in a cluster
    \item Multiprocessor systems
    \item Multicore systems
    \item In-cloud and Edge computing
\end{itemize}

\subsection{MPI Program Structure}
\begin{minted}{c}
#include "mpi.h"
#include <stdio.h>
#include <stdlib.h>

int main(int argc, char *argv[]) {
    int numtasks, rank, dest, source, rc, count, tag=1;
    char inmsg, outmsg='x';
    MPI_Status Stat;
    
    MPI_Init(&argc, &argv);
    MPI_Comm_size(MPI_COMM_WORLD, &numtasks);
    MPI_Comm_rank(MPI_COMM_WORLD, &rank);
    
    if (rank == 0) {
        dest = 1; source = 1;
        rc = MPI_Send(&outmsg, 1, MPI_CHAR, dest, tag, MPI_COMM_WORLD);
        rc = MPI_Recv(&inmsg, 1, MPI_CHAR, source, tag, MPI_COMM_WORLD, &Stat);
    }
    else if (rank == 1) {
        dest = 0; source = 0;
        rc = MPI_Recv(&inmsg, 1, MPI_CHAR, source, tag, MPI_COMM_WORLD, &Stat);
        rc = MPI_Send(&outmsg, 1, MPI_CHAR, dest, tag, MPI_COMM_WORLD);
    }
    
    MPI_Get_count(&Stat, MPI_CHAR, &count);
    printf("Task %d: Received %d char(s) from task %d with tag %d\n", 
           rank, count, Stat.MPI_SOURCE, Stat.MPI_TAG);
    
    MPI_Finalize();
}
\end{minted}

\textcolor{blue}{[Summary: Basic MPI programs initialize with MPI\_Init, get process count and rank, perform communication operations, and finalize with MPI\_Finalize. Processes are identified by unique ranks within communicators.]}

\section{MPI Communication Fundamentals}

\subsection{Basic Send and Receive Operations}
\begin{itemize}
    \item \textbf{MPI\_Send(void *sendbuf, int nelems, int dest, int tag, MPI\_Comm comm)}
    \item \textbf{MPI\_Recv(void *recvbuf, int nelems, int source, int tag, MPI\_Comm comm, MPI\_Status *status)}
    \item Parameters:
          \begin{itemize}
              \item sendbuf/recvbuf: pointers to data buffers
              \item nelems: number of data elements
              \item dest/source: process identifiers
              \item tag: message identifier (non-negative integer)
              \item comm: communicator (usually MPI\_COMM\_WORLD)
          \end{itemize}
\end{itemize}

\subsection{Communicators and Groups}
\begin{itemize}
    \item \textbf{Communicators}: Define which processes can communicate
    \item \textbf{MPI\_COMM\_WORLD}: Predefined communicator including all processes
    \item \textbf{Rank}: Unique integer identifier for each process within a communicator
    \item Ranks are contiguous starting from 0
    \item Used to specify source/destination of messages and control program flow
\end{itemize}

\textcolor{blue}{[Summary: MPI uses communicators to define communication groups, with ranks identifying individual processes for message routing and conditional execution.]}

\section{Point-to-Point Communication}

\subsection{Communication Types}
\begin{itemize}
    \item Synchronous send
    \item Blocking send/blocking receive
    \item Non-blocking send/non-blocking receive
    \item Buffered send
    \item Combined send/receive
    \item "Ready" send
    \item Any send type can pair with any receive type
\end{itemize}

\subsection{System Buffers}
\begin{itemize}
    \item Buffer space reserved for data in transit
    \item Managed entirely by MPI library (opaque to programmer)
    \item Finite resource that can be exhausted
    \item Can exist on sending side, receiving side, or both
    \item Allows send-receive operations to be asynchronous
\end{itemize}

\textcolor{teal}{[Concept Map: Point-to-Point Communication → Blocking/Non-blocking → Synchronous/Asynchronous → Buffered/Non-buffered → System Buffer Management]}

\subsection{Blocking vs Non-blocking Operations}

\subsubsection{Blocking Operations}
\begin{itemize}
    \item \textbf{Blocking Send}: Returns only when safe to modify application buffer
    \item \textbf{Blocking Receive}: Returns only when data has arrived and is ready
    \item Can be synchronous (handshaking) or asynchronous (using system buffers)
\end{itemize}

\subsubsection{Non-blocking Operations}
\begin{itemize}
    \item Return immediately without waiting for communication events
    \item Simply "request" MPI to perform operation when able
    \item Unsafe to modify application buffer until operation completes
    \item Use "wait" routines (MPI\_Wait, MPI\_Test) to check completion
    \item Used to overlap computation with communication
\end{itemize}

\textcolor{orange}{[Mnemonic: Blocking = Wait for completion, Non-blocking = Fire and (maybe) forget until checked]}

\section{Collective Communication}

\subsection{Types of Collective Operations}
\begin{itemize}
    \item \textbf{Synchronization}: Processes wait until all reach synchronization point (MPI\_Barrier, MPI\_Waitall)
    \item \textbf{Data Movement}: Broadcast, scatter/gather, all to all
    \item \textbf{Collective Computation (Reductions)}: One member collects data and performs operations (min, max, add, multiply)
\end{itemize}

\subsection{Important Collective Routines}

\subsubsection{Broadcast (MPI\_Bcast)}
\begin{itemize}
    \item Broadcasts message from one task to all others in communicator
    \item \texttt{MPI\_Bcast(\&buffer, count, datatype, root, comm)}
\end{itemize}

\subsubsection{Scatter (MPI\_Scatter)}
\begin{itemize}
    \item Sends chunks of data from one task to all tasks
    \item \texttt{MPI\_Scatter (sendbuf, sendcnt, sendtype, recvbuf, recvcnt, recvtype, root, comm)}
\end{itemize}

\subsubsection{Gather (MPI\_Gather)}
\begin{itemize}
    \item Gathers data from all tasks to a single task
    \item Inverse of scatter operation
\end{itemize}

\subsubsection{Reduce (MPI\_Reduce)}
\begin{itemize}
    \item Performs reduction (sum, max, min, etc.) across all tasks
    \item Result stored in one task
    \item \texttt{MPI\_Reduce(sendbuf, recvbuf, count, datatype, op, root, comm)}
\end{itemize}

\subsubsection{Allreduce (MPI\_Allreduce)}
\begin{itemize}
    \item Performs reduction and stores result across all tasks
\end{itemize}

\textcolor{blue}{[Summary: Collective operations involve all processes in a communicator for synchronization, data movement, or collective computation, with routines like broadcast, scatter/gather, and reductions.]}

\section{Derived Data Types}

\subsection{Continuous Derived Data Type}
\begin{itemize}
    \item Represents contiguous blocks of data
    \item \texttt{MPI\_Type\_contiguous(count, oldtype, \&newtype)}
    \item Example: Representing a row of a 2D array
\end{itemize}

\subsection{Vector Derived Data Type}
\begin{itemize}
    \item Represents regularly spaced data blocks
    \item \texttt{MPI\_Type\_vector(count, blocklength, stride, oldtype, \&newtype)}
    \item Example: Representing a column of a 2D array
\end{itemize}

\subsection{Indexed Derived Data Type}
\begin{itemize}
    \item Represents irregular data patterns
    \item \texttt{MPI\_Type\_indexed(count, array\_of\_blocklengths, array\_of\_displacements, oldtype, \&newtype)}
\end{itemize}

\subsection{Struct Derived Data Type}
\begin{itemize}
    \item Represents heterogeneous data structures
    \item \texttt{MPI\_Type\_struct(count, array\_of\_blocklengths, array\_of\_offsets, array\_of\_types, \&newtype)}
    \item Example: Representing a particle with position, velocity, and type information
\end{itemize}

\textcolor{orange}{[Mnemonic: Continuous = Contiguous blocks, Vector = Regular spacing, Indexed = Irregular patterns, Struct = Mixed data types]}

\section{Groups and Communicators}

\subsection{Definitions}
\begin{itemize}
    \item \textbf{Group}: Ordered set of processes with unique integer ranks (0 to N-1)
    \item \textbf{Communicator}: Encompasses a group of processes that can communicate
    \item \textbf{MPI\_COMM\_WORLD}: Default communicator containing all processes
    \item Groups and communicators are dynamic - can be created/destroyed during execution
    \item Processes can be in multiple groups/communicators with different ranks
\end{itemize}

\subsection{Group Management Operations}
\begin{enumerate}
    \item Extract global group: \texttt{MPI\_Comm\_group(MPI\_COMM\_WORLD, \&group)}
    \item Create new group: \texttt{MPI\_Group\_incl(group, n, ranks, \&newgroup)}
    \item Create new communicator: \texttt{MPI\_Comm\_create(comm, group, \&newcomm)}
    \item Get new rank: \texttt{MPI\_Comm\_rank(newcomm, \&rank)}
    \item Free resources: \texttt{MPI\_Comm\_free(), MPI\_Group\_free()}
\end{enumerate}

\subsection{Communicator Splitting}
\begin{itemize}
    \item \texttt{MPI\_Comm\_split(comm, color, key, \&newcomm)}
    \item Groups processes by color, sorts by key within each color group
    \item Useful for organizing processes by function or data partitioning
\end{itemize}

\textcolor{blue}{[Summary: Groups define process sets, communicators enable communication within those sets, and MPI provides operations to dynamically create and manage these entities for flexible parallel programming.]}

\section{Virtual Topologies}

\subsection{Concept and Purpose}
\begin{itemize}
    \item Virtual topology: Mapping/ordering of MPI processes into geometric shapes
    \item Types: Cartesian (grid) and Graph topologies
    \item Virtual - no relation to physical machine structure required
    \item Must be programmed by application developer
\end{itemize}

\subsection{Benefits}
\begin{itemize}
    \item \textbf{Convenience}: Match communication patterns to application needs
    \item \textbf{Efficiency}: Potential optimization for specific hardware architectures
    \item Example: Cartesian topology for grid-based computations with nearest-neighbor communication
\end{itemize}

\subsection{Cartesian Topology Example}
\begin{itemize}
    \item Processes arranged in 2D grid with row-major ordering
    \item Each process identified by (row, column) coordinates
    \item Supports operations like \texttt{MPI\_Cart\_sub()} for creating sub-grids
\end{itemize}

\textcolor{teal}{[Concept Map: Virtual Topologies → Cartesian/Graph Types → Process Mapping → Communication Patterns → Potential Hardware Optimization]}

\section{MPI Programming Examples}

\subsection{Monte Carlo Pi Calculation}

\subsubsection{Mathematical Basis}
\begin{itemize}
    \item Circle area: $\pi r^2$, Square area: $4r^2$
    \item Ratio: $\frac{\pi r^2}{4r^2} = \frac{\pi}{4}$
    \item Random points: $M \approx \frac{N\pi}{4}$
    \item Approximation: $\pi \approx \frac{4M}{N}$
\end{itemize}

\subsubsection{Parallel Implementation Strategy}
\begin{itemize}
    \item Divide total points among available tasks
    \item Each task calculates points in circle for its subset
    \item Master task collects results using send/receive operations
    \item No data dependencies between tasks
\end{itemize}

\subsection{Nearest Neighbor Exchange}
\begin{minted}{c}
// Non-blocking ring communication
MPI_Irecv(&buf[0], 1, MPI_INT, prev, tag1, MPI_COMM_WORLD, &reqs[0]);
MPI_Irecv(&buf[1], 1, MPI_INT, next, tag2, MPI_COMM_WORLD, &reqs[1]);
MPI_Isend(&rank, 1, MPI_INT, prev, tag2, MPI_COMM_WORLD, &reqs[2]);
MPI_Isend(&rank, 1, MPI_INT, next, tag1, MPI_COMM_WORLD, &reqs[3]);

// Do work while communications progress
MPI_Waitall(4, reqs, stats);
\end{minted}

\textcolor{blue}{[Summary: MPI enables various parallel algorithms through point-to-point and collective communication, with examples including Monte Carlo simulations and nearest-neighbor exchanges in ring topologies.]}

\section{MPI Advantages and Characteristics}

\subsection{Key Features}
\begin{itemize}
    \item \textbf{Practical}: Easy to implement on any system
    \item \textbf{Portable}: Works on networks, clusters, multiprocessors
    \item \textbf{Efficient}: Programmer has more control over communication
    \item \textbf{Flexible}: Easy to scale problem and processor size
    \item \textbf{Explicit Parallelism}: Burden on programmer to identify and manage parallelism
\end{itemize}

\section{Study Aids and Exam Preparation}

\subsection*{Key Concepts to Master}
\begin{itemize}
    \item Difference between blocking and non-blocking operations
    \item Purpose and usage of communicators and groups
    \item Collective communication operations and when to use them
    \item Derived data types and their applications
    \item Virtual topologies and their benefits
    \item MPI program structure and initialization/finalization
\end{itemize}

\subsection*{Exam Questions}
\textcolor{red}{Exam Questions:}
\begin{enumerate}
    \item Explain the difference between blocking and non-blocking communication in MPI, including when you would use each type.
    \item Describe how MPI communicators and groups work, and provide an example of when you might create a new communicator.
    \item Write pseudocode for a parallel Monte Carlo $\pi$ calculation using MPI collective operations.
    \item What are derived data types in MPI and why are they useful? Provide an example.
    \item Explain the concept of virtual topologies in MPI and discuss potential benefits for parallel applications.
\end{enumerate}
\textcolor{red}{]}

\subsection*{Common MPI Functions Reference}
\begin{itemize}
    \item Environment: MPI\_Init, MPI\_Finalize, MPI\_Comm\_size, MPI\_Comm\_rank
    \item Point-to-point: MPI\_Send, MPI\_Recv, MPI\_Isend, MPI\_Irecv, MPI\_Wait
    \item Collective: MPI\_Bcast, MPI\_Scatter, MPI\_Gather, MPI\_Reduce, MPI\_Barrier
    \item Groups/Communicators: MPI\_Comm\_split, MPI\_Comm\_create, MPI\_Group\_incl
\end{itemize}

\textcolor{orange}{[Mnemonic: MPI Basics - Init, Size, Rank, Send, Recv, Finalize. Collective Ops - Bcast, Scatter, Gather, Reduce. Advanced - Groups, Topologies, Derived Types]}

\end{document}
